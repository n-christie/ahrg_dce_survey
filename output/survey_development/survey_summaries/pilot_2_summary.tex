% Options for packages loaded elsewhere
\PassOptionsToPackage{unicode}{hyperref}
\PassOptionsToPackage{hyphens}{url}
%
\documentclass[
]{article}
\usepackage{amsmath,amssymb}
\usepackage{iftex}
\ifPDFTeX
  \usepackage[T1]{fontenc}
  \usepackage[utf8]{inputenc}
  \usepackage{textcomp} % provide euro and other symbols
\else % if luatex or xetex
  \usepackage{unicode-math} % this also loads fontspec
  \defaultfontfeatures{Scale=MatchLowercase}
  \defaultfontfeatures[\rmfamily]{Ligatures=TeX,Scale=1}
\fi
\usepackage{lmodern}
\ifPDFTeX\else
  % xetex/luatex font selection
\fi
% Use upquote if available, for straight quotes in verbatim environments
\IfFileExists{upquote.sty}{\usepackage{upquote}}{}
\IfFileExists{microtype.sty}{% use microtype if available
  \usepackage[]{microtype}
  \UseMicrotypeSet[protrusion]{basicmath} % disable protrusion for tt fonts
}{}
\makeatletter
\@ifundefined{KOMAClassName}{% if non-KOMA class
  \IfFileExists{parskip.sty}{%
    \usepackage{parskip}
  }{% else
    \setlength{\parindent}{0pt}
    \setlength{\parskip}{6pt plus 2pt minus 1pt}}
}{% if KOMA class
  \KOMAoptions{parskip=half}}
\makeatother
\usepackage{xcolor}
\usepackage[margin=1in]{geometry}
\usepackage{graphicx}
\makeatletter
\def\maxwidth{\ifdim\Gin@nat@width>\linewidth\linewidth\else\Gin@nat@width\fi}
\def\maxheight{\ifdim\Gin@nat@height>\textheight\textheight\else\Gin@nat@height\fi}
\makeatother
% Scale images if necessary, so that they will not overflow the page
% margins by default, and it is still possible to overwrite the defaults
% using explicit options in \includegraphics[width, height, ...]{}
\setkeys{Gin}{width=\maxwidth,height=\maxheight,keepaspectratio}
% Set default figure placement to htbp
\makeatletter
\def\fps@figure{htbp}
\makeatother
\setlength{\emergencystretch}{3em} % prevent overfull lines
\providecommand{\tightlist}{%
  \setlength{\itemsep}{0pt}\setlength{\parskip}{0pt}}
\setcounter{secnumdepth}{-\maxdimen} % remove section numbering
\usepackage{booktabs}
\usepackage{longtable}
\usepackage{array}
\usepackage{multirow}
\usepackage{wrapfig}
\usepackage{float}
\usepackage{colortbl}
\usepackage{pdflscape}
\usepackage{tabu}
\usepackage{threeparttable}
\usepackage{threeparttablex}
\usepackage[normalem]{ulem}
\usepackage{makecell}
\usepackage{xcolor}
\usepackage{multicol}
\usepackage{hhline}
\newlength\Oldarrayrulewidth
\newlength\Oldtabcolsep
\usepackage{hyperref}
\ifLuaTeX
  \usepackage{selnolig}  % disable illegal ligatures
\fi
\IfFileExists{bookmark.sty}{\usepackage{bookmark}}{\usepackage{hyperref}}
\IfFileExists{xurl.sty}{\usepackage{xurl}}{} % add URL line breaks if available
\urlstyle{same}
\hypersetup{
  pdftitle={Pilot \#2 summary},
  hidelinks,
  pdfcreator={LaTeX via pandoc}}

\title{Pilot \#2 summary}
\author{}
\date{\vspace{-2.5em}2024-07-05}

\begin{document}
\maketitle

\hypertarget{second-pilot-summary}{%
\subsection{Second pilot summary}\label{second-pilot-summary}}

The second pilot of the DCE experiment was conducted with the
\emph{Intresspoolen} panel with 32 individuals participating. A number
of changes were made to the initial survey which was administered among
mostly researchers at Lund. The following changed were made in this
pilot test:

\begin{itemize}
\tightlist
\item
  The attribute levels were incorrectly specified in the first pilot
  test. This was corrected in the second pilot.
\item
  The survey text was shortened to reduce the time to complete the
  survey.
\item
  Language was refined and reduced to make the survey more easily
  interpretable.
\item
  Grammatical mistakes resulting from English-Swedish translations were
  corrected and phrasing was rewritten to make the text more clear.
\end{itemize}

Respondents were sent a link to the experiment by email.

The experiment is hosted on our private server and connected to our
research project domain name:

\url{https://www.reloc-age.org/dce}

The survey is powered by the opensource survey framework Formr.

\hypertarget{descriptive-data}{%
\subsection{Descriptive data}\label{descriptive-data}}

32 individuals from the \emph{Intresspoolen} participated in the second
pilot. The average monthly housing cost for the participants was 9090
SEK while the average household income was 45900 SEK. The average
planned cost for prospective housing among the the participants was 9380
SEK. The average time to complete the DCE experiment was 11.3 minutes,
while the average time to complete the feedback section was 3.17
minutes. Regarding user experience and content, the participants were
generally satisfied with an average score of 3.81 out of 5 for user
experience and 3.93 out 5 for content.

\global\setlength{\Oldarrayrulewidth}{\arrayrulewidth}

\global\setlength{\Oldtabcolsep}{\tabcolsep}

\setlength{\tabcolsep}{0pt}

\renewcommand*{\arraystretch}{1.5}



\providecommand{\ascline}[3]{\noalign{\global\arrayrulewidth #1}\arrayrulecolor[HTML]{#2}\cline{#3}}

\begin{longtable}[c]{|p{2.35in}|p{1.52in}}

\caption{Sample\ description}\\

\ascline{1.5pt}{666666}{1-2}

\multicolumn{1}{>{\raggedright}m{\dimexpr 2.35in+0\tabcolsep}}{\textcolor[HTML]{000000}{\fontsize{11}{11}\selectfont{ }}} & \multicolumn{1}{>{\centering}m{\dimexpr 1.52in+0\tabcolsep}}{\textcolor[HTML]{000000}{\fontsize{11}{11}\selectfont{Overall}}\textcolor[HTML]{000000}{\fontsize{11}{11}\selectfont{\linebreak }}\textcolor[HTML]{000000}{\fontsize{11}{11}\selectfont{(N=\ \ 32)}}} \\

\ascline{1.5pt}{666666}{1-2}\endfirsthead \caption[]{Sample\ description}\\

\ascline{1.5pt}{666666}{1-2}

\multicolumn{1}{>{\raggedright}m{\dimexpr 2.35in+0\tabcolsep}}{\textcolor[HTML]{000000}{\fontsize{11}{11}\selectfont{ }}} & \multicolumn{1}{>{\centering}m{\dimexpr 1.52in+0\tabcolsep}}{\textcolor[HTML]{000000}{\fontsize{11}{11}\selectfont{Overall}}\textcolor[HTML]{000000}{\fontsize{11}{11}\selectfont{\linebreak }}\textcolor[HTML]{000000}{\fontsize{11}{11}\selectfont{(N=\ \ 32)}}} \\

\ascline{1.5pt}{666666}{1-2}\endhead



\multicolumn{1}{>{\raggedright}m{\dimexpr 2.35in+0\tabcolsep}}{\textcolor[HTML]{000000}{\fontsize{11}{11}\selectfont{\textbf{Monthly\ costs}}}} & \multicolumn{1}{>{\centering}m{\dimexpr 1.52in+0\tabcolsep}}{\textcolor[HTML]{000000}{\fontsize{11}{11}\selectfont{}}} \\





\multicolumn{1}{>{\raggedright}m{\dimexpr 2.35in+0\tabcolsep}}{\textcolor[HTML]{000000}{\fontsize{11}{11}\selectfont{  Mean\ (SD)}}} & \multicolumn{1}{>{\centering}m{\dimexpr 1.52in+0\tabcolsep}}{\textcolor[HTML]{000000}{\fontsize{11}{11}\selectfont{9090\ (5820)}}} \\





\multicolumn{1}{>{\raggedright}m{\dimexpr 2.35in+0\tabcolsep}}{\textcolor[HTML]{000000}{\fontsize{11}{11}\selectfont{  Median\ [Min,\ Max]}}} & \multicolumn{1}{>{\centering}m{\dimexpr 1.52in+0\tabcolsep}}{\textcolor[HTML]{000000}{\fontsize{11}{11}\selectfont{8000\ [0,\ 25000]}}} \\





\multicolumn{1}{>{\raggedright}m{\dimexpr 2.35in+0\tabcolsep}}{\textcolor[HTML]{000000}{\fontsize{11}{11}\selectfont{\textbf{Monthly\ household\ income}}}} & \multicolumn{1}{>{\centering}m{\dimexpr 1.52in+0\tabcolsep}}{\textcolor[HTML]{000000}{\fontsize{11}{11}\selectfont{}}} \\





\multicolumn{1}{>{\raggedright}m{\dimexpr 2.35in+0\tabcolsep}}{\textcolor[HTML]{000000}{\fontsize{11}{11}\selectfont{  Mean\ (SD)}}} & \multicolumn{1}{>{\centering}m{\dimexpr 1.52in+0\tabcolsep}}{\textcolor[HTML]{000000}{\fontsize{11}{11}\selectfont{45900\ (37800)}}} \\





\multicolumn{1}{>{\raggedright}m{\dimexpr 2.35in+0\tabcolsep}}{\textcolor[HTML]{000000}{\fontsize{11}{11}\selectfont{  Median\ [Min,\ Max]}}} & \multicolumn{1}{>{\centering}m{\dimexpr 1.52in+0\tabcolsep}}{\textcolor[HTML]{000000}{\fontsize{11}{11}\selectfont{38800\ [0,\ 125000]}}} \\





\multicolumn{1}{>{\raggedright}m{\dimexpr 2.35in+0\tabcolsep}}{\textcolor[HTML]{000000}{\fontsize{11}{11}\selectfont{\textbf{Planned\ costs}}}} & \multicolumn{1}{>{\centering}m{\dimexpr 1.52in+0\tabcolsep}}{\textcolor[HTML]{000000}{\fontsize{11}{11}\selectfont{}}} \\





\multicolumn{1}{>{\raggedright}m{\dimexpr 2.35in+0\tabcolsep}}{\textcolor[HTML]{000000}{\fontsize{11}{11}\selectfont{  Mean\ (SD)}}} & \multicolumn{1}{>{\centering}m{\dimexpr 1.52in+0\tabcolsep}}{\textcolor[HTML]{000000}{\fontsize{11}{11}\selectfont{9380\ (4860)}}} \\





\multicolumn{1}{>{\raggedright}m{\dimexpr 2.35in+0\tabcolsep}}{\textcolor[HTML]{000000}{\fontsize{11}{11}\selectfont{  Median\ [Min,\ Max]}}} & \multicolumn{1}{>{\centering}m{\dimexpr 1.52in+0\tabcolsep}}{\textcolor[HTML]{000000}{\fontsize{11}{11}\selectfont{10000\ [0,\ 23000]}}} \\





\multicolumn{1}{>{\raggedright}m{\dimexpr 2.35in+0\tabcolsep}}{\textcolor[HTML]{000000}{\fontsize{11}{11}\selectfont{\textbf{Time\ to\ complete\ DCE\ section}}}} & \multicolumn{1}{>{\centering}m{\dimexpr 1.52in+0\tabcolsep}}{\textcolor[HTML]{000000}{\fontsize{11}{11}\selectfont{}}} \\





\multicolumn{1}{>{\raggedright}m{\dimexpr 2.35in+0\tabcolsep}}{\textcolor[HTML]{000000}{\fontsize{11}{11}\selectfont{  Mean\ (SD)}}} & \multicolumn{1}{>{\centering}m{\dimexpr 1.52in+0\tabcolsep}}{\textcolor[HTML]{000000}{\fontsize{11}{11}\selectfont{11.3\ (24.9)}}} \\





\multicolumn{1}{>{\raggedright}m{\dimexpr 2.35in+0\tabcolsep}}{\textcolor[HTML]{000000}{\fontsize{11}{11}\selectfont{  Median\ [Min,\ Max]}}} & \multicolumn{1}{>{\centering}m{\dimexpr 1.52in+0\tabcolsep}}{\textcolor[HTML]{000000}{\fontsize{11}{11}\selectfont{6.57\ [0.433,\ 146]}}} \\





\multicolumn{1}{>{\raggedright}m{\dimexpr 2.35in+0\tabcolsep}}{\textcolor[HTML]{000000}{\fontsize{11}{11}\selectfont{\textbf{Time\ to\ complete\ feedback}}}} & \multicolumn{1}{>{\centering}m{\dimexpr 1.52in+0\tabcolsep}}{\textcolor[HTML]{000000}{\fontsize{11}{11}\selectfont{}}} \\





\multicolumn{1}{>{\raggedright}m{\dimexpr 2.35in+0\tabcolsep}}{\textcolor[HTML]{000000}{\fontsize{11}{11}\selectfont{  Mean\ (SD)}}} & \multicolumn{1}{>{\centering}m{\dimexpr 1.52in+0\tabcolsep}}{\textcolor[HTML]{000000}{\fontsize{11}{11}\selectfont{3.17\ (3.27)}}} \\





\multicolumn{1}{>{\raggedright}m{\dimexpr 2.35in+0\tabcolsep}}{\textcolor[HTML]{000000}{\fontsize{11}{11}\selectfont{  Median\ [Min,\ Max]}}} & \multicolumn{1}{>{\centering}m{\dimexpr 1.52in+0\tabcolsep}}{\textcolor[HTML]{000000}{\fontsize{11}{11}\selectfont{2.21\ [0.267,\ 12.7]}}} \\





\multicolumn{1}{>{\raggedright}m{\dimexpr 2.35in+0\tabcolsep}}{\textcolor[HTML]{000000}{\fontsize{11}{11}\selectfont{  Missing}}} & \multicolumn{1}{>{\centering}m{\dimexpr 1.52in+0\tabcolsep}}{\textcolor[HTML]{000000}{\fontsize{11}{11}\selectfont{2\ (6.3\%)}}} \\





\multicolumn{1}{>{\raggedright}m{\dimexpr 2.35in+0\tabcolsep}}{\textcolor[HTML]{000000}{\fontsize{11}{11}\selectfont{\textbf{User\ Experience}}}} & \multicolumn{1}{>{\centering}m{\dimexpr 1.52in+0\tabcolsep}}{\textcolor[HTML]{000000}{\fontsize{11}{11}\selectfont{}}} \\





\multicolumn{1}{>{\raggedright}m{\dimexpr 2.35in+0\tabcolsep}}{\textcolor[HTML]{000000}{\fontsize{11}{11}\selectfont{  Mean\ (SD)}}} & \multicolumn{1}{>{\centering}m{\dimexpr 1.52in+0\tabcolsep}}{\textcolor[HTML]{000000}{\fontsize{11}{11}\selectfont{3.81\ (1.05)}}} \\





\multicolumn{1}{>{\raggedright}m{\dimexpr 2.35in+0\tabcolsep}}{\textcolor[HTML]{000000}{\fontsize{11}{11}\selectfont{  Median\ [Min,\ Max]}}} & \multicolumn{1}{>{\centering}m{\dimexpr 1.52in+0\tabcolsep}}{\textcolor[HTML]{000000}{\fontsize{11}{11}\selectfont{4.00\ [1.00,\ 5.00]}}} \\





\multicolumn{1}{>{\raggedright}m{\dimexpr 2.35in+0\tabcolsep}}{\textcolor[HTML]{000000}{\fontsize{11}{11}\selectfont{  Missing}}} & \multicolumn{1}{>{\centering}m{\dimexpr 1.52in+0\tabcolsep}}{\textcolor[HTML]{000000}{\fontsize{11}{11}\selectfont{1\ (3.1\%)}}} \\





\multicolumn{1}{>{\raggedright}m{\dimexpr 2.35in+0\tabcolsep}}{\textcolor[HTML]{000000}{\fontsize{11}{11}\selectfont{\textbf{Content}}}} & \multicolumn{1}{>{\centering}m{\dimexpr 1.52in+0\tabcolsep}}{\textcolor[HTML]{000000}{\fontsize{11}{11}\selectfont{}}} \\





\multicolumn{1}{>{\raggedright}m{\dimexpr 2.35in+0\tabcolsep}}{\textcolor[HTML]{000000}{\fontsize{11}{11}\selectfont{  Mean\ (SD)}}} & \multicolumn{1}{>{\centering}m{\dimexpr 1.52in+0\tabcolsep}}{\textcolor[HTML]{000000}{\fontsize{11}{11}\selectfont{3.93\ (1.05)}}} \\





\multicolumn{1}{>{\raggedright}m{\dimexpr 2.35in+0\tabcolsep}}{\textcolor[HTML]{000000}{\fontsize{11}{11}\selectfont{  Median\ [Min,\ Max]}}} & \multicolumn{1}{>{\centering}m{\dimexpr 1.52in+0\tabcolsep}}{\textcolor[HTML]{000000}{\fontsize{11}{11}\selectfont{4.00\ [2.00,\ 5.00]}}} \\





\multicolumn{1}{>{\raggedright}m{\dimexpr 2.35in+0\tabcolsep}}{\textcolor[HTML]{000000}{\fontsize{11}{11}\selectfont{  Missing}}} & \multicolumn{1}{>{\centering}m{\dimexpr 1.52in+0\tabcolsep}}{\textcolor[HTML]{000000}{\fontsize{11}{11}\selectfont{2\ (6.3\%)}}} \\

\ascline{1.5pt}{666666}{1-2}



\end{longtable}



\arrayrulecolor[HTML]{000000}

\global\setlength{\arrayrulewidth}{\Oldarrayrulewidth}

\global\setlength{\tabcolsep}{\Oldtabcolsep}

\renewcommand*{\arraystretch}{1}

\hypertarget{regression-analysis}{%
\subsection{Regression analysis}\label{regression-analysis}}

While only 32 individuals participated in the second pilot, the data
collected from this DCE experiment (32 individuals x 9 choice sets = 288
obs) is sufficient to estimate the preferences of the participants for
the attributes of the housing alternatives.

Data results from the survey is analysed using standard methods found in
the literature where the utility that an individual assigns to a
particular choice alternative is estimated using a conditional logit
model based on random utility theory. The utility of an alternative is
represented as:

\begin{equation}
U_{ij} = V_{ij} + \varepsilon_{ij}
\end{equation}

\noindent where \(U_{ij}\) is the total utility that respondent
\textit{i} associates with choice alternative \textit{j} and \(V_{ij}\)
is the systematic utility, which is a function of the attributes of the
alternative and the preferences of the respondent. \(\varepsilon_{ij}\)
is the random error term, representing unobserved factors and
individual-specific preferences. The systematic utility, \(V_{ij}\), is
modelled as a linear function of attribute levels:

\begin{equation}
V_{ij} = \beta_1 X_{ij1} + \beta_2 X_{ij2} + \ldots + \beta_k X_{ijk}
\end{equation}

\noindent where \(\beta_1\), \(\beta_2\), \(\ldots\), \(\beta_k\) are
the coefficients representing the part-worth utilities of the attribute
levels and \(X_{ij1}\), \(X_{ij2}\), \(\ldots\), \(X_{ijk}\) are the
levels of the attributes for choice alternative \textit{j} in scenario
\textit{i}.

\vspace{2mm}

\noindent The probability that respondent \(i\) chooses alternative
\(j\) from a set of alternatives is modeled using the choice probability
as follows:

\begin{equation}
P_{ij} = \frac{e^{V_{ij}}}{\sum_{l=1}^{J} e^{V_{il}}}
\end{equation}

\noindent where \(P_{ij}\) is the probability that respondent \textit{i}
chooses alternative \textit{j}. \textit{J} is the total number of
alternatives in the choice set. To estimate the coefficients (\(\beta\)
values) in the systematic utility equation, maximum likelihood
estimation is used. The likelihood function for the conditional logit
model is maximized to find the best-fitting coefficients that maximize
the probability of observing the actual choices made by respondents
given the attributes of the alternatives.

Each attribute was coded as a factor variable with the base level set to
minimum value for the locational attributes. The base level for the
price attribute was set at zero to estimate the effect of bi directional
price changes on the choice of housing.

Below are the results:

Logistic regression results

~

Model 1

Distance green - 5km (base = 500 meters)

-0.88 {[}-1.38; -0.38{]}

Distance green - 15km (base = 500 meters)

-1.09 {[}-1.57; -0.62{]}

Distance shops - 5km (base = 500 meters)

-0.76 {[}-1.27; -0.25{]}

Distance shops - 15km (base = 500 meters)

-1.26 {[}-1.78; -0.75{]}

Distance trans - 200m (base = 200m)

0.08 {[}-0.51; 0.67{]}

Distance trans - 800m (base = 200m)

-0.02 {[}-0.66; 0.63{]}

Parking - Garageplats (base = Ingen p-plats)

0.84 {[} 0.25; 1.44{]}

Parking - P-plats (base = Ingen p-plats)

0.90 {[} 0.33; 1.47{]}

Price -20\% (base = 0)

0.35 {[}-0.26; 0.96{]}

Price -10\% (base = 0)

0.21 {[}-0.38; 0.80{]}

Price 10\% (base = 0)

-0.42 {[}-1.05; 0.20{]}

Price 20\% (base = 0)

-0.85 {[}-1.51; -0.20{]}

Log Likelihood

-152.51

null.logLik

-199.63

AIC

329.03

BIC

372.98

R2

0.24

Adj. R2

0.18

nobs

288.00

* 0 outside the confidence interval.

From the results of the regression analysis, we can see that the
distance to green areas, distance to shops, and parking availability all
have a significant effects in comparison to their base levels. All
significant coefficients demonstrate the expected signs with negative
coefficients, representing a decrease in utility as the distance
increases, observed for the distance to green areas and distance to
shops. Distance to public transport has no significant effect on the
choice of housing. Compared to no available parking place, both garage
and parking place have a positive effect on the choice of housing. On
average, respondents prefer being Regarding price, the results show that
a 20\% increase in price has a negative effect on the choice of housing,
while all other price levels have no significant effect on the choice of
housing.

\hypertarget{comments}{%
\subsection{Comments}\label{comments}}

Beleow are the comments from the feedback section of the survey.
Respondents were asked to provide feedback on the user experience and
content of the survey.

\hypertarget{user-experience}{%
\subsubsection{User experience}\label{user-experience}}

\begin{longtable}[t]{l}
\caption{\label{tab:unnamed-chunk-1}}\\
\toprule
Comments on content\\
\midrule
Jeg manglede spørgsmål om boligens udformning (lejlighed, have etc..) samt boform (lejlighed, hus, seniorbolig etc)\\
Vill bo centralt.\\
Dom är relevanta men, om jag förstått undersökningen rätt, tycker jag undersökningen är alldeles för komplicerad för "vanligt" folk.\\
Tillgänglighetsaspekter på fastigheten bör definitivt vara med.\\
Saknar avstånd till arbetsplats\\
\addlinespace
Det är just de här olika typerna av bostäders placering kontra olika servicetyper som är viktiga att ta ställning till inför eventuell flytt.\\
Lite väl långt till service alternativt till grönområde i förhållande till kostnadsnivå.\\
Det är ok med de bostadspreferenser som ingår i undersökningen när de följs upp och man kan följa om de förändras med stigande ålder. Grönomtåde, parkeringsplats och en lägenhet med uteplats gärna centralt är ju vad man vill ha😉 |\\
Jag Reflekterar vi mycket högre grad kring avståndet Till grönområde. Jag upplever detta som viktigare än avstånd till affär eller möjligheten att ha reserverat p-plats.\\
Boendekostnaden är viktig utifrån att man har en bestämd summa att förhålla sig till. Sedan är boendemiljön och kollektivtrafiken viktig för att vardagen ska fungera.\\
\addlinespace
Åker inte kollektivt, pga handikapp, så den parametern går bort\\
Jag är pensionär , man borde ha fått frågan om ålder t ex\\
Många andra viktiga preferenser saknas, t ex utbud av kultur, föreningsliv, fritid, estetiskt tilltalande miljö, närhet till släkt och vänner.\\
Grönområden fins i närheten 
Dåligt med lägenheter däremot, finns dyra nybyggda bostadsrätter 
Så inte riktigt applicerbart på oss\\
Tycker det är relevanta bostadspreferenser. Det jag själv också försöker tänka på också är om det är i en större stad eller ute på landsbygd som dök upp när jag försökte finna vilket svar som var relevant för mig.\\
\bottomrule
\end{longtable}

\hypertarget{content}{%
\subsubsection{Content}\label{content}}

\begin{longtable}[t]{l}
\caption{\label{tab:unnamed-chunk-2}}\\
\toprule
User experience comments\\
\midrule
Alldeles för komplicerad för att "vanliga" människor ska ta sig igenom den. De flesta kommer att ge upp efter 2-3 frågor eftersom det är svårt att se skillnaden mellan alternativen. Dessutom förekommer ett språk som endast välutbildade människor är familjära med, t.ex. "...info om preferenserna som ingår i undersökningen". På något ställe står också att ett syfte är att "optimera den slutliga versionen". En annan sak jag hänger upp mig på i den senare delen är att jag inte kan gå tillbaka och se vad man svarat, utan att jag måste komma ihåg det.\\
Alternativet "vägrar flytta" borde vara med där inget av de två alternativen kan accepteras.\\
Tankeväckande alternativ vad gäller bostadsalternativen, jag fick verkligen tänka till.\\
lätt att genomföra men behöver man verkligen komma ihåg detaljerna när det gäller preferenserna?\\
Min upplevelse var att jag ofta valde mellan pest eller kolera.\\
\addlinespace
Bra upplägg, en för mig viktig sak i enkäten var att det skall finnas en reserverad plats för bilen\\
Kan vara svårare att svara på undersökningen på mobil.\\
Helt Ok frågor\\
Bra att behöva tänka till om jag skulle behöva ett annat boende framöver. Tänka efter vad det är jag tycker är viktigt för att klara mig själv så länge som möjligt.\\
Stimulerande, tänkvärd\\
\addlinespace
Finns ingenting om ev bostadstillägg.. heller ingenting om man är ensam eller ett par.\\
Vi bor på landet usel kolektivtrafik och helt beroende avvbil\\
\bottomrule
\end{longtable}

\hypertarget{conclusion}{%
\subsection{Conclusion}\label{conclusion}}

After receiving the results from the second pilot, a number of
considerations have arisen which should be addressed and discussed to
develop our experiment further.

\begin{itemize}
\item
  Firstly, while the sample size of 32 individuals is much smaller than
  our anticipated respondent base of perhaps 600 individuals, we are
  observing significant effects in the regression analysis. This is a
  positive sign that the DCE experiment is working as intended and that
  the preferences of the respondents are being captured by the
  experiment as we designed it. The number of choice sets per individual
  (9) and the nuberm of attributes and levels should be more than
  sufficient to estimate the preferences of the respondents in our
  target population. \textbf{In fact, we should be able to comfortably
  add more attributes/levels to the experiment to capture more of the
  preferences of the respondents.}
\item
  Secondly, the regression results demonstrate that further attention to
  the attribute levels is needed.

  \begin{itemize}
  \tightlist
  \item
    There are not many surprises in our results - it is very expected
    that respondents gain more utility from living closer to amenities.
  \item
    How can we make the results more interpretable and relevant? -
    \textbf{Standardize the levels and create equal intervals for the
    attributes.}
  \item
    Since we used the same levels for distance to shops and green areas,
    we can compare the coefficient sizes. We can't compare these
    attributes to distance to transportation in any meaningful way.
  \item
    \textbf{We should standardize the levels of the attributes to make
    the results more interpretable and to allow for more meaningful
    comparisons between the attributes}
  \item
  \end{itemize}
\item
  Lastly, the feedback suggests the levels of the attributes may not be
  applicable to everyone as some people live in rural areas and some
  urban. This highlights the need for our presentation of the experiment
  as a hypothetical scenario. The scenarios we present do not need to be
  ``one size fits all'' and should be simplified to make it easier for
  the respondents, as well as make our results more interpretable
  \textbf{We should make it clear that the housing choices are
  hypothetical and that the respondent should consider the attributes as
  if they were real.}

  \begin{itemize}
  \tightlist
  \item
    An example of this is to simplify the presentation of the different
    hypothetical apartments as being located in the city center and
    changing the attribute distances to be in meters rather than
    kilometers. Our experiment is by nature hypothetical and does not
    desire any preconditions to identify preferences.
  \end{itemize}
\end{itemize}

\end{document}
