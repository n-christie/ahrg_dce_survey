\documentclass[3p,11pt ]{elsarticle}

\makeatletter
\def\ps@pprintTitle{%
 \let\@oddhead\@empty
 \let\@evenhead\@empty
 \let\@oddfoot\@empty
 \let\@evenfoot\@empty
}
\makeatother


%% Use the option review to obtain double line spacing
%% \documentclass[preprint,review,12pt]{elsarticle}

%% Use the options 1p,twocolumn; 3p; 3p,twocolumn; 5p; or 5p,twocolumn
%% for a journal layout:
%% \documentclass[final,1p,times]{elsarticle}
%% \documentclass[final,1p,times,twocolumn]{elsarticle}
%% \documentclass[final,3p,times]{elsarticle}
%% \documentclass[final,3p,times,twocolumn]{elsarticle}
%% \documentclass[final,5p,times]{elsarticle}
%% \documentclass[final,5p,times,twocolumn]{elsarticle}

%% For including figures, graphicx.sty has been loaded in
%% elsarticle.cls. If you prefer to use the old commands
%% please give \usepackage{epsfig}
%% The amssymb package provides various useful mathematical symbols



\usepackage{natbib}
 \bibpunct[, ]{(}{)}{,}{a}{}{,}%
 \def\bibfont{\small}%
 \def\bibsep{\smallskipamount}%
 \def\bibhang{24pt}%
 \def\newblock{\ }%
 \def\BIBand{and}%


\usepackage{lipsum} 
\usepackage{amssymb}
\usepackage{amsmath}
\usepackage{booktabs}
\usepackage{longtable}
\usepackage{array}
\usepackage{multirow}
\usepackage{wrapfig}
\usepackage{float}
\usepackage{colortbl}
\usepackage{pdflscape}
\usepackage{tabu}
\usepackage{threeparttable}
\usepackage{threeparttablex}
\usepackage[normalem]{ulem}
\usepackage{makecell}
\usepackage{xcolor}

\usepackage{booktabs}   % for top/mid/bottomrule
\usepackage{dcolumn}    % for D column alignment
\usepackage{caption}    % optional, for better caption formatting
\usepackage{amsmath}    % optional, for symbols like chi²

%\usepackage[nomarkers]{endfloat}
\usepackage{setspace}
\usepackage{graphicx}
\usepackage{float}
\usepackage{rotating}
\usepackage{hyperref}
\hypersetup{
  colorlinks=true,
  linkcolor=blue}
\usepackage{caption}
\captionsetup{font=normalsize}
% mini-tabular inside a cell to align around the comma
\newcommand{\ci}[2]{\begin{tabular}{@{}r@{}l@{}}(#1,& #2)\end{tabular}}

%% The amsthm package provides extended theorem environments
%% \usepackage{amsthm}

%% The lineno packages adds line numbers. Start line numbering with
%% \begin{linenumbers}, end it with \end{linenumbers}. Or switch it on
%% for the whole article with \linenumbers.
%% \usepackage{lineno}

\journal{Housing studies}

\begin{document}

\begin{frontmatter}

%% Title, authors and addresses

%% use the tnoteref command within \title for footnotes;
%% use the tnotetext command for theassociated footnote;
%% use the fnref command within \author or \address for footnotes;
%% use the fntext command for theassociated footnote;
%% use the corref command within \author for corresponding author footnotes;
%% use the cortext command for theassociated footnote;
%% use the ead command for the email address,
%% and the form \ead[url] for the home page:
%% \title{Title\tnoteref{label1}}
%% \tnotetext[label1]{}
%% \author{Name\corref{cor1}\fnref{label2}}
%% \ead{email address}
%% \ead[url]{home page}
%% \fntext[label2]{}
%% \cortext[cor1]{}
%% \affiliation{organization={},
%%             addressline={},
%%             city={},
%%             postcode={},
%%             state={},
%%             country={}}
%% \fntext[label3]{}

\title{The Value of Location: What Matters Most for Older Individuals Considering Relocation in Sweden?}

%% use optional labels to link authors explicitly to addresses:
%% \author[label1,label2]{}
%% \affiliation[label1]{organization={},
%%             addressline={},
%%             city={},
%%             postcode={},
%%             state={},
%%             country={}}
%%
%% \affiliation[label2]{organization={},
%%             addressline={},
%%             city={},
%%             postcode={},
%%             state={},
%%             country={}}

\author[1]{Nick Christie}
\ead{nick.christie@med.lu.se}

\author[1]{Bj\"orn Slaug}
\ead{bjorn.slaug@med.lu.se}

\author[1]{Magnus Zingmark}
\ead{magnus.zingmark@med.lu.se}

\author[2]{Jonas Bj\"ork}
\ead{jonas.bjork@med.lu.se}


\author[1]{Susanne Iwarsson}
\ead{susanne.iwarsson@med.lu.se}



%\author[1]{Maya Ky\'eln}
%\ead{maya.kyeln@med.lu.se}






%\author[1]{Steven M Schmidt}
%\ead{steven.schmidt@med.lu.se}















 \affiliation[1]{organization={Department of Health Sciences, Lund University}, 
                 addressline={P.O. Box 7080},
                 postcode={22100}, 
                 city={Lund}, 
                 country={Sweden}}
                 
 \affiliation[2]{organization={Division of Occupational and Environmental Medicine, Lund University}, 
                 addressline={Scheelevägen 2},
                 postcode={22363}, 
                 city={Lund}, 
                 country={Sweden}}
                 
                 
%\begin{abstract}
%
%This study explores variation in housing attribute preferences among older adults in Sweden considering relocation in Sweden.
%Using a discrete choice experimental data from the Prospective RELOC-AGE project (n=957;mean age = 71.9;55.3\% women), key housing attributes including proximity to services, access to public transportation, availability of green space, and presence of dedicated parking facilities were examined. These attributes are framed using percentage-based trade-offs to estimate marginal willingness to pay for each feature. Socio-demographic factors such as age and gender are included to identify systematic differences in preferences. The results reveal substantial heterogeneity. Individuals in the oldest age groups express significantly higher willingness to pay for several attributes, with some values reaching up to three times those of younger respondents. Differences between men and women, renters and owners, are also observed. The findings provide evidence-based estimates that support policy development and housing planning aimed at facilitating ageing in place and improving the design of environments that accommodate diverse needs in later life.
%\end{abstract}


\begin{abstract}
We examined heterogeneity in housing preferences among older adults in Sweden using discrete choice experiment data (n = 957; mean age = 72; 55.3\% women). Respondents assessed trade-offs between key residential attributes, including proximity to shops and services, green spaces, access
to public transportation, dedicated parking, and planned monthly expenses. We estimate mixed logit models to recover marginal willingness to pay estimates for each attribute, including interactions with age, gender, and income to capture systematic variation in preferences. The results show that individuals in the oldest age groups expressed significantly higher willingness to pay for several attributes, with values around 1.5 times greater than those of younger respondents. We identified meaningful differences by gender and tenure status, reflecting underlying patterns of social inequality in later life. These findings contribute policy-relevant evidence to support the development of age-inclusive housing strategies that address both diverse preferences and structural disparities in residential choice. 

\end{abstract}
%\begin{keyword}
%
%Ageing \sep Housing \sep Health \sep Relocation
%%% keywords here, in the form: keyword \sep keyword
%
%%% PACS codes here, in the form: \PACS code \sep code
%
%%% MSC codes here, in the form: \MSC code \sep code
%%% or \MSC[2008] code \sep code (2000 is the default)
%
%\end{keyword}
\end{frontmatter}

%% \linenumbers

\newpage



\section{Introduction}

As populations age and the expansion of housing associated with boundless population growth maintains pace,
understanding the housing preferences of the ageing demographic becomes increasingly important for future societies.
It is well known that societies across the globe are facing an increasing proportion of older individuals.
Research suggests that most older adults prefer to age in place,
changing households to a lesser extent compared to younger demographics  \citep{abramssonChangingLocationsCentral2015}.
In Sweden, the majority of senior citizens live in their own homes with 94\% of the population aged 65+ remaining in ordinary housing \citep{jennbertDevelopmentsElderlyPolicy2009}.
As this population segment grows,
appropriate housing options are essential to provide viable options for an ageing society in need of housing.
When relocations occur,
what matters most for older individuals in their pursuit of accommodation becomes essential for understanding our future needs as a society.

This study employs a stated choice experiment to delve into the critical factors influencing the housing choices of older individuals in Sweden who are considering relocation.
%Using key housing attributes identified in the nation-wide Prospective RELOC-AGE project,
%we are able to take a closer look at the many factors which matter most for this segment.
Our tests use a diverse sample of older individuals across Sweden considering relocation, presenting them with hypothetical scenarios that vary in locational attributes, including proximity to green spaces, access to public transportation, shops, and parking availability.

We calculate marginal willingness to pay estimates,
allowing a monetary interpretation and providing valuable insights for rural planners, policy-makers, and healthcare providers, offering guidance on creating age-friendly environments that may cater to the unique needs and desires of older home adults in their communities.


This study is related to a number of research areas.
First,
we contribute to the growing literature related to housing and ageing.
Second,
on a methodological level, we contribute to the literature involving stated choice experiments.
To the best of our knowledge,
our study is the first to examine housing preferences in the Swedish context.
Lastly,
we contribute to ageing in place research by identifying differences in housing preferences among older adults.
With a larger proportions of older adults in most parts of the world,
knowledge into what may influence independence at home is valuable to governments, policy setters, urban planners, and private companies.
We contribute to this literature with evidence towards preferred housing attributes of older adults.


In a similar study,
\cite{ossokinaBestLivingConcepts2020} 
run a discrete choice experiment utilizing predominately building characteristics in their model.
In the existing literature,
\cite{ossokinaReferencedependentHousingChoice2022a} is closest to our study in the choice of locational attributes.
The authors find that proximity to public transport has the highest effect on utility in their sample, followed by proximity to shops.
Our paper differs from theirs in a few ways.
First,
our sample is larger an encompasses a larger age group.
They have 441 home-owners in a nine year age group (65-75).
Second,
our study comprises both home owners and rental occupants, representing a larger diversity of individuals searching for housing.
Finally,
instead of a purely hypothetical experiment, the respondents in our study have signed up for housing services, where we can assume they have the intention, or at least a strong consideration, to relocate.





\section{Method}

\subsection{Participants}

This paper utilizes survey and experiment data derived from the Prospective RELOC-AGE project,
a longitudinal two-tiered mixed-method cohort investigation conducted in Sweden.
The study was registered under the identifier NCT04765696 on ClinicalTrials.gov (U.S. National Library of Medicine, 2021).
\footnote{For comprehensive insights into the study's procedures,
please refer to the study protocol  \cite{zingmarkExploringAssociationsHousing2021}}
Data collection from this study was in conjunction with the second follow-up survey administered in May 2024.
A geographical diverse sample of individuals aged 55 and above was recruited for this study across Sweden (see Figure \ref{fig:map}).

The primary objective of the Prospective RELOC-AGE study was to explore the long-term dynamics associated with housing choices, relocation, and active and healthy ageing, focusing on individuals across the ageing process.
%In Sweden, approximately 4-5\% of individuals aged 60-84 years relocate each year (Statistics Sweden, 2020).
Eligible participants were individuals aged 55 or older, residing in Sweden, and actively registered for relocation with one of three housing companies: two local public housing providers and a national provider of tenant-owned dwellings.

\begin{wrapfigure}[22]{r}{0.3\textwidth}
\centering
\includegraphics[scale=0.5]{figures/survey_location.png}
\caption{Distribution of respondents \label{fig:map}}
\end{wrapfigure}


\subsection{Ethics}

The Prospective RELOC-AGE study was approved by the Swedish Ethical Review Authority (No. 2020–03457), in alignment with the Declaration of Helsinki and current national ethical regulations for research involving human participants.
Potential participants received written information highlighting that participation was voluntary and that declining would not affect their access to housing options or public services.
Informed consent was considered given upon the completion and return of the survey.



\subsection{Experiment design and attributes}

Stated choice experiments include a range of techniques in which respondents indicate their preferences by explicitly stating their choices.
In contrast to revealed choice experiments, where preferences are inferred from past behaviour, stated choice methods allow for the evaluation of decisions in a controlled environment.
This controlled setting enables the researcher to systematically manipulate attributes and isolate the impact of specific factors on decision-making.
A Discrete Choice Experiment (DCE) is a type of stated choice model that presents individuals with hypothetical scenarios, allowing researchers to quantify how much value respondents place on different attributes of a product, service, or housing option.

DCE's have been used in a variety of housing and ageing studies.
\cite{ossokinaBestLivingConcepts2020} estimate a stated choice experiment to study the residential preferences of elderly homewoners in the Netherlands,
reporting that residential attributes connecting to safety and social cohesion play an important role for elderly.
\cite{ossokinaReferencedependentHousingChoice2022a}


Respondents were asked to choose the most desirable housing option from a set of alternatives which contained varying levels of attributes.
Each respondent was then presented with multiple scenarios, or choice sets,
where each choice set contained random attribute levels.
To reduce the dimensionality of a full factorial design, a D-optimal subset was generated that reduces dimensionality while maintaining statistical power.


The choice of attributes and associated levels in our study was a combination of factors identified from the Prospective RELOC-AGE follow-up study and attributes found in the housing literature.
Proximity to green areas has been examined in numerous contexts including improved cardiometabolic and general health \citep{paquetAreAccessibilityCharacteristics2013,  maasGreenSpaceUrbanity2006},
lower stress \citep{nielsenGreenAreasAffect2007},
and improved mental health \citep{cohen-clineAccessGreenSpace2015,sturmProximityUrbanParks2014}.
Proximity to shops and services represent not only distance to frequent ammenities which may become more burdensome to transverse,
but also also may constitute an integral social experience to participate in the social live of communities
\citep{lucasMethodEvaluateEquitable2016}.
Distance to transportation has been shown to affect  accessibility levels of populations,
with significant differences identified in senior cohorts \citep{ricciardiExploringPublicTransport2015,hildebrandDimensionsElderlyTravel2003,alsnihMobilityAccessibilityExpectations2003}
Available parking facilities may also affect acceptability,
particularly for our sample where over 90\% of respondents indicated access to an automobile.

We select levels that allow for interpretability and viability of our willingness to pay estimates while holding certain attributes constant to make meaningful comparisons \citep{hensherAppliedChoiceAnalysis2015}.
The attribute \textit{greenspace} is defined as the distance in kilometres to green areas including parks, forests, hiking areas, and open spaces.
Similarly,
\textit{shops} represents the distance to shopping amenities such as grocery stores, malls, boutiques, and shopping centres.
The attribute \textit{transport} is the distance to transportation, such as a bus stop, metro station, or train station.
\textit{price} represents the percentage change from respondent's planned housing cost in 10\% intervals.
Table \ref{tab:atts}  shows the attributes and their corresponding levels used in the experiment.


\begin{table}[H]

\caption{Attributes and descriptions \label{tab:atts}}
\centering
\begin{tabular}[top]{>{\raggedright\arraybackslash}p{15em}l}
\toprule
Attribute & Description and levels\\
\midrule
Distance to green area & \makecell[l]{1 = within 500m \\ 2 = within 10km \\ 3 = within 15km}\\
\addlinespace
Distance to shops & \makecell[l]{1 = within 500m \\ 2 = within 10km \\ 3 = within 15km}\\
\addlinespace
Distance to public transportation & \makecell[l]{1 = within 300m \\ 2 = within 600 \\ 3 = within 900m}\\
\addlinespace
Parking & \makecell[l]{1 = no reserved parking \\ 2 = reserved parking place \\ 3 = reserved garage place}\\
\addlinespace
Price & \makecell[l]{1 = 20 \% less than planned costs \\ 2 = 10 \% less than planned costs \\ 3 = same as planned cost \\ 4 = 10 \% more than planned costs  \\ 5 = 20 \% more than planned costs}\\
\bottomrule
\end{tabular}
\end{table}


Before commencing the experiment,
respondents where given a definition of each attribute, as well as an example to clarify any ambiguity in interpretation of the attributes.
Respondents were also instructed to base each choice on the assumption that the alternative housing options where identical in every way aside from the attribute levels.
Figure \ref{fig:choice_set} depicts a typical choice set presented to the respondents.


\begin{figure}[H]
\centering
\includegraphics[scale=0.50]{figures/choice_set.png}
\caption{Example choice set \label{fig:choice_set}}
\end{figure}

The number of choice sets was limited to 9 in order to minimize the cognitive burden of the DCE while maximizing the statistical power of our tests \citep{manghamHowNotDesigning2009,deshazoDesigningChoiceSets2002}
\footnote{\cite{himmlerWhatWorksBetter2021} highlights increased age would tend to exacerbated the cognitive burden of a discrete choice experiment, suggesting complex designs would lead to unreliable results.}.




%Despite the large amounts of attention these attributes have seen in the context of ageing and housing,
%knowledge gaps still remain.
%For instance,
%it is common that older individuals are examined as one homogeneous group in comparison to, for example, a younger cohort.
%This simplified grouping may miss key differences and shortcomings in interpretations.
%Importantly,
%this heterogeneity may be intensified throughout the population with the growth of the older population in the coming years.





%
%\cite{ricciardiExploringPublicTransport2015} "explore accessibility levels to public transport systems in the context of Perth, Australia, comparing seniors, low-income populations, populations without car availability, and the rest of the population. The findings reveal that the biggest accessibility differences are found in the seniors group, showing the lowest accessibility level to public transport systems."
%
%
%However, some studies have found significant differences within the senior cohort regarding travel behaviour and capacity to access certain locations.
%\cite{hildebrandDimensionsElderlyTravel2003}
%\cite{alsnihMobilityAccessibilityExpectations2003}

%\citep{lucasMethodEvaluateEquitable2016}. stress that physical grocery shopping may act as an integrated social experience.  Proximity to such shopping amenities can be crucial for older individuals to participate in the social life of their communities. 

%\cite{cohen-clineAccessGreenSpace2015} exaimine mental health and green areas among adult twin pairs, finding that greater access to green space is associated with less depression.
%
%In a Dutch study, \cite{maasGreenSpaceUrbanity2006} find that increased green space significantly reduces perceived general health, with elderly populations particularly benefiting more from the presence of green areas in their living environment.
%
%Using Danish survey data\cite{nielsenGreenAreasAffect2007} report results that indicate access to green areas are associtated with lower stress.
%
%\cite{paquetAreAccessibilityCharacteristics2013} report a relatinship between open spaces and better cardiometabolic health.
%
%
%\cite{sturmProximityUrbanParks2014} report findings that mental health is significantly related to residential distance from parks, suggesting mental health could be improved with enhanced environmental interventions. 





\subsection{Development and data}

Prior to administering the DCE, we conducted and internal review and one pilot study (n = 20; n = 56).
The internal review involved researchers and staff within the University’s network.
The second pilot was administered to participants in Lund University’s Intressentpoolen,
a network of individuals placed in collaboration with academia and society in the areas of ageing and health\footnote{ For more information see: https://www.case.lu.se/intressentpoolen}.
In both studies, respondents completed the experiment and a structured feedback form assessing the relevance of attributes, clarity of wording and levels, and overall task comprehension and burden.
Feedback informed several revisions: clarifying attribute descriptions and levels; removing redundant instructional text; fine-tuning attribute ranges; and reducing the number of choice sets from 11 to 9 to manage respondent burden. Pilot data were also used to confirm that the planned sample size would provide adequate statistical power with stable parameter estimates.

The choice experiment was administered in conjunction with the follow-up Prospective RELOC-AGE survey in May 2024,
where 1,247 respondents agreed to participate in the survey via a web-based platform.
Following the initial survey administration, 
two reminder mails were sent out to encouraged participation in the DCE.
The first in September 2024,
to the entire respondent group,
and another reminder in October 2024,
targeting only those who had not taken the experiment portion of the survey.
This resulted in a 80\% increase from initial turn out.
The final sample size of 957 individuals was finalized, representing a 73\% response rate.
%The experiment portion of the survey was administered within the flexible survey framework Formr,
%which was hosted on university servers.

Table \ref{tab:desc} provides descriptive statistics for the study sample, disaggregated by housing tenure.
Of the 957 respondents, 790 were homeowners and 167 were renters.
The sample was slightly majority female (55.3\%), with most participants aged between 65–74 years (43.5\%) followed by those aged 75+ (36.5\%).
The majority of respondents were partnered (60.6\%), retired (77.1\%), and lived in apartment or condominium housing (63.0\%).
Most had attained at least three years of university education (47.6\%), and lived in city or town locations (62.2\%).
Households predominantly consisted of one or two members (95.2\%), with only 3.8\% reporting three or more.
The mean monthly household income was 46,700 SEK, and the average planned monthly housing cost was 10,500 SEK.



\begin{table}[H]
\centering
\footnotesize
\begin{threeparttable}
\caption{\label{tab:desc}Descriptive statistics}
\begin{tabular}[t]{lrrr}
\toprule
 & Owner & Renter & Overall \\
\midrule
 & (N = 790) & (N = 167) & (N = 957) \\
\addlinespace[0.8em]

Sex & & & \\
\hspace{1em}Female & 419 (53.0\%) & 110 (65.9\%) & 529 (55.3\%) \\
\hspace{1em}Male & 371 (47.0\%) & 57 (34.1\%) & 428 (44.7\%) \\
\addlinespace[0.6em]

Age group & & & \\
\hspace{1em}55--64 & 154 (19.5\%) & 38 (22.8\%) & 192 (20.1\%) \\
\hspace{1em}65--74 & 344 (43.5\%) & 72 (43.1\%) & 416 (43.5\%) \\
\hspace{1em}75+ & 292 (37.0\%) & 57 (34.1\%) & 349 (36.5\%) \\
\addlinespace[0.6em]

Civil status & & & \\
\hspace{1em}Not partnered & 287 (36.3\%) & 90 (53.9\%) & 377 (39.4\%) \\
\hspace{1em}Partnered & 503 (63.7\%) & 77 (46.1\%) & 580 (60.6\%) \\
\addlinespace[0.6em]

Education & & & \\
\hspace{1em}Elementary school & 42 (5.3\%) & 17 (10.2\%) & 59 (6.2\%) \\
\hspace{1em}2 years upper secondary & 68 (8.6\%) & 17 (10.2\%) & 85 (8.9\%) \\
\hspace{1em}3 or 4 years upper secondary & 127 (16.1\%) & 30 (18.0\%) & 157 (16.4\%) \\
\hspace{1em}University $<$ 3 years & 161 (20.4\%) & 37 (22.2\%) & 198 (20.7\%) \\
\hspace{1em}University $\geq$ 3 years & 390 (49.4\%) & 66 (39.5\%) & 456 (47.6\%) \\
\addlinespace[0.6em]

Retired & & & \\
\hspace{1em}Retired & 612 (77.5\%) & 126 (75.4\%) & 738 (77.1\%) \\
\hspace{1em}Not retired & 174 (22.0\%) & 41 (24.6\%) & 215 (22.5\%) \\
\addlinespace[0.6em]

Current housing type & & & \\
\hspace{1em}Apartment/Condo & 449 (56.8\%) & 154 (92.2\%) & 603 (63.0\%) \\
\hspace{1em}House & 341 (43.2\%) & 13 (7.8\%) & 354 (37.0\%) \\
\addlinespace[0.6em]

Current housing location & & & \\
\hspace{1em}City/town & 478 (60.5\%) & 117 (70.1\%) & 595 (62.2\%) \\
\hspace{1em}Urban area & 221 (28.0\%) & 41 (24.6\%) & 262 (27.4\%) \\
\hspace{1em}Countryside & 81 (10.3\%) & 6 (3.6\%) & 87 (9.1\%) \\
\addlinespace[0.6em]

Number in household & & & \\
\hspace{1em}1 inhabitant& 240 (30.4\%) & 81 (48.5\%) & 321 (33.5\%) \\
\hspace{1em}2 inhabitants & 509 (64.4\%) & 81 (48.5\%) & 590 (61.7\%) \\
\hspace{1em}3 or more inhabitants& 34 (4.3\%) & 2 (1.2\%) & 36 (3.8\%) \\
\addlinespace[0.6em]

Monthly household income & 48,700 (34,700) & 37,700 (27,900) & 46,700 (33,900) \\
Planned housing costs & 10,400 (4,060) & 10,500 (5,570) & 10,500 (4,370) \\
\bottomrule
\end{tabular}

\begin{tablenotes}[flushleft]
\scriptsize
\item \textit{Note:} Table reports descriptive statistics of variable in the study. Values are frequencies with percentages in parentheses. Monthly household income and planned housing costs are reported as mean (standard deviation) in SEK (Swedish Krona).
\end{tablenotes}
\end{threeparttable}
\end{table}


\clearpage




\subsection{Statistical analyses}

Utilizing the discrete choice data,
respondent's choices may be modelled within a random utility theory framework,
which assumes individuals choose options which maximize their utility based upon available options \citep{lancsarConductingDiscreteChoice2008}.
An underlying utility model can then be estimated where the utility that individual \( i \) derives from alternative \( j \) in a choice set \( t \) is given by:

\begin{equation}
U_{ijt} = V_{ijt} + \varepsilon_{ijt}
\end{equation}

\noindent where \( V_{ijt} \) is the systematic component of utility, modelled as a function of the attributes of the alternative, and \( \varepsilon_{ijt} \) is an unobserved random error term.
The systematic utility is specified as:

\begin{equation}
V_{ijt} = \beta_1 X_{ijt1} + \beta_2 X_{ijt2} + \ldots + \beta_k X_{ijtk}
\end{equation}

\noindent where \( X_{ijtk} \) represents the level of attribute \( k \) for alternative \( j \) in task \( t \), and \( \beta_k \) are the corresponding utility coefficients to be estimated.
Here,
\( X_{ijtk} \) represent the respective attributes found in Table \ref{tab:atts}.

While multinomial logit (MNL) models,
which assume homogeneous preferences across respondents and independently and identically distributed (i.i.d.) error terms,
are commonly estimated in choice modelling,
they are problematic when the objective is to uncover heterogeneity in preferences, as in our study. 
Recent research addresses this limitation by allowing utility coefficients to vary across individuals or groups \citep{aitkenOlderHomebuyersPrefer2024, zhaoUsingConjointAnalysis2023, caplanMeasuringHeterogeneousPreferences2021}. 
We follow this line of research and estimate mixed logit (ML) models, in which the utility coefficients \( \boldsymbol{\beta}_i \) are allowed to vary randomly across individuals to account for unobserved heterogeneity in preferences \citep{mcfaddenMixedMNLModels2000}:


\begin{equation}
\boldsymbol{\beta}_i = \boldsymbol{\beta} + \Sigma \eta_i
\end{equation}

\noindent where \( \eta_i \sim \mathcal{N}(0, I) \) and \( \Sigma \) is the covariance matrix of the random parameters. 
The mixed logit model captures heterogeneity by integrating over the distribution of random coefficients using simulated maximum likelihood. 
Because this integral has no closed-form solution, we approximate it using simulation:

\begin{equation}
\mathcal{L}^S(\boldsymbol{\beta}, \Sigma) = \sum_{i=1}^{N} 
\ln \left[ \frac{1}{R} \sum_{r=1}^{R} 
P_i(y_i \mid \boldsymbol{\beta}_i^{r}) \right]
\end{equation}

\noindent where each simulated draw \( \boldsymbol{\beta}_i^{r} = \boldsymbol{\beta} + \Sigma \eta_i^{r} \), and \( R \) denotes the number of Sobol draws used to approximate the integral over the distribution of random parameters \citep{trainDiscreteChoiceMethods2003}. 
In our specification, all non-monetary attributes are treated as random and normally distributed, while the price coefficient is fixed to preserve consistency in the computation of willingness-to-pay (WTP) measures. 
We also allow for full covariance among the random parameters to capture potential correlations in preferences.

We then use our price coefficient to compute monetary trade-offs for non-cost attributes. 
First,
we estimate marginal rate of substitution (MRS) for each attribute,
where the MRS for attribute \( k \) is calculated as:

\begin{equation}
\text{MRS}_k = -\frac{\hat{\beta}_k}{\hat{\beta}_{\text{cost}}}
\label{eq:mrs}
\end{equation}

\noindent where \( \hat{\beta}_{\text{cost}} \) is the estimated coefficient on the cost attribute.
Because the cost attribute is specified as a percentage change from the respondent’s expected housing cost in 10 percent intervals, we convert marginal rates of substitution (MRS) into marginal willingness to pay (MWTP) in monetary terms by scaling with 10 percent of the mean reported monthly housing cost:

\begin{equation}
\text{MWTP}_k = \text{MRS}_k \times 0.10 \times \bar{PC}
\label{eq:mwtp}
\end{equation}

\noindent where $\bar{PC}$ denotes the mean of respondents' stated planned monthly housing cost.
These MWTP estimates represent the amount, in SEK per month, that respondents are willing to pay for improvements in each housing attribute, relative to their baseline housing cost expectations. 

All models are estimated in R using the \texttt{logitr} package \citep{helvestonLogitrFastEstimation2023}, which provides flexible estimation routines for both MNL and ML models using simulated maximum likelihood.
For the ML models,
we use 100 Sobol draws and multiple random starting values to ensure convergence to the global maximum \citep{trainDiscreteChoiceMethods2003}.




\section{Empirical Results}

Our discussion begins with results from the baseline mixed logit (MXL) models, which exclude interaction terms and serve as a point of comparison for subsequent models that account for heterogeneity across household types. Tables \ref{tab:base_owner} and \ref{tab:base_renter} present the estimation results for respondents who own and rent their housing units, respectively. In both samples, positive (negative) coefficients indicate an increase (decrease) in average utility associated with the attribute level, relative to its reference level. Reference levels are specified as the least desirable alternative to aid in interpretation.

\begin{table}[!htbp]
\caption{Base Specification - Owners}
\begin{center}
\begin{scriptsize}
\begin{threeparttable}

% --- define \ci macro for centered-on-comma CIs ---
\newcommand{\ci}[2]{\begin{tabular}{@{}r@{}l@{}}(#1,& #2)\end{tabular}}

\begin{tabular}{l D{.}{.}{4.5} D{.}{.}{2.5} c D{.}{.}{3.2}}
\toprule
 & \multicolumn{1}{c}{MXL - \(\mu\)} 
 & \multicolumn{1}{c}{MXL - SD} 
 & \multicolumn{1}{c}{MRS} 
 & \multicolumn{1}{c}{MWTP} \\
\midrule

Green space: 5 km (vs 15 km)       & 1.16^{***}  & 0.12        & 0.22                          & 232.29 \\
                                   & (0.12)      & (0.23)      & \multicolumn{1}{c}{\ci{0.17}{0.27}} &        \\
Green space: 500 m (vs 15 km)      & 2.21^{***}  & 1.24^{***}  & 0.42                         & 440.66 \\
                                   & (0.16)      & (0.25)      & \multicolumn{1}{c}{\ci{0.35}{0.49}} &        \\
Shops: 5 km (vs 15 km)             & 1.01^{***}  & 0.10        & 0.19                         & 200.82 \\
                                   & (0.13)      & (0.20)      & \multicolumn{1}{c}{\ci{0.13}{0.25}} &        \\
Shops: 500 m (vs 15 km)            & 3.12^{***}  & 1.05^{***}  & 0.59                          & 622.34 \\
                                   & (0.18)      & (0.21)      & \multicolumn{1}{c}{\ci{0.51}{0.67}} &        \\
Transit stop: 600 m (vs 900 m)     & 0.35^{**}   & -0.76^{***} & 0.07                         & 69.56  \\
                                   & (0.11)      & (0.20)      & \multicolumn{1}{c}{\ci{0.02}{0.11}} &        \\
Transit stop: 300 m (vs 900 m)     & 1.13^{***}  & 0.79^{***}  & 0.21                         & 225.03 \\
                                   & (0.12)      & (0.18)      & \multicolumn{1}{c}{\ci{0.17}{0.26}} &        \\
Parking: reserved garage (vs none) & 3.08^{***}  & 0.56^{*}    & 0.59                          & 615.13 \\
                                   & (0.20)      & (0.25)      & \multicolumn{1}{c}{\ci{0.50}{0.67}} &        \\
Parking: reserved space (vs none)  & 2.56^{***}  & -2.29^{***} & 0.49                          & 511.17 \\
                                   & (0.17)      & (0.17)      & \multicolumn{1}{c}{\ci{0.42}{0.55}} &        \\
Price                              & -5.26^{***} &             &                                    &        \\
                                   & (0.35)      &             &                                    &        \\

\midrule
Num. obs.          & 7110    &             &        &        \\
Log Likelihood     & -3136.51 &            &        &        \\
AIC                & 6363.02 &             &        &        \\
BIC                & 6672.14 &             &        &        \\
McFadden R\textsuperscript{2} & 0.36    &        &        &        \\
LR $\chi^{2}$ (df=9)          & 3583.53 &        &        &        \\
p-value (LR)       & 0.00    &             &        &        \\
\bottomrule

\end{tabular}

\begin{tablenotes}
\scriptsize
\item \textit{Notes:} Estimation by maximum likelihood of the multinomial logit (MNL) and mixed logit (ML) models. 
Heteroskedasticity-robust standard errors are reported in parentheses. 
Marginal rate of substitution (MRS) and marginal willingness to pay (MWTP) estimates are computed from the ML model following the procedures described in the text. 
Ninety-five percent confidence intervals for MRS estimates are obtained using the delta method. 
$^{*}p < 0.1$; $^{**}p < 0.05$; $^{***}p < 0.01$.
\end{tablenotes}

\end{threeparttable}
\end{scriptsize}
\label{tab:base_owner}
\end{center}
\end{table}


Across both tenure groups, all estimated coefficients exhibit the expected signs and a high degree of statistical significance. For most attributes, the mean coefficients are positive, implying that closer proximity to amenities or improved neighborhood features are preferred over their reference levels. The statistically significant standard deviation estimates further confirm the presence of unobserved preference heterogeneity among households, supporting the use of the mixed logit specification.

The marginal rates of substitution (MRS) are reported in column five and are computed as the ratio of attribute to price coefficients, following Equation (\ref{eq:mrs}). The corresponding marginal willingness to pay (MWTP) values in column six are obtained by scaling the MRS by 10 percent of the average planned housing cost for each tenure group, as described in Equation (\ref{eq:mwtp}).

As shown in Table \ref{tab:base_owner}, homeowners are willing to pay substantial amounts for proximity to key neighborhood amenities. For instance, owners are, on average, willing to pay about 440 SEK per month to avoid living 15 km from the nearest green area and roughly 620 SEK per month to live within 500 m of shops. Access to parking is also highly valued, with MWTP estimates of approximately 615 SEK for a reserved garage and 511 SEK for a reserved parking space. These results suggest that accessibility and convenience are primary drivers of residential utility among homeowners.

\begin{table}[!htbp]
\caption{Base Specification - Renters}
\begin{center}
\begin{scriptsize}
\begin{threeparttable}

% --- define \ci macro for centered-on-comma CIs ---
\newcommand{\ci}[2]{\begin{tabular}{@{}r@{}l@{}}(#1,& #2)\end{tabular}}

\begin{tabular}{l D{.}{.}{4.5} D{.}{.}{2.5} c D{.}{.}{3.2}}
\toprule
 & \multicolumn{1}{c}{MXL - \(\mu\)} 
 & \multicolumn{1}{c}{MXL - SD} 
 & \multicolumn{1}{c}{MRS} 
 & \multicolumn{1}{c}{MWTP} \\
\midrule

Green space: 5 km (vs 15 km)       & 4.80^{***}   & 2.30^{***}  & 0.36                            & 377.54 \\
                                   & (1.29)       & (0.65)      & \multicolumn{1}{c}{\ci{0.24}{0.48}} &        \\
Green space: 500 m (vs 15 km)      & 7.85^{***}   & 6.64^{***}  & 0.59                            & 618.10 \\
                                   & (2.07)       & (1.85)      & \multicolumn{1}{c}{\ci{0.42}{0.75}} &        \\
Shops: 5 km (vs 15 km)             & 3.67^{**}    & 0.36        & 0.27                            & 288.71 \\
                                   & (1.15)       & (0.46)      & \multicolumn{1}{c}{\ci{0.15}{0.40}} &        \\
Shops: 500 m (vs 15 km)            & 8.09^{***}   & 6.47^{***}  & 0.61                            & 636.95 \\
                                   & (2.12)       & (1.81)      & \multicolumn{1}{c}{\ci{0.44}{0.78}} &        \\
Transit stop: 600 m (vs 900 m)     & 1.57^{**}    & 0.99        & 0.12                            & 123.73 \\
                                   & (0.55)       & (0.63)      & \multicolumn{1}{c}{\ci{0.05}{0.19}} &        \\
Transit stop: 300 m (vs 900 m)     & 2.01^{***}   & 1.64^{**}   & 0.15                            & 158.21 \\
                                   & (0.52)       & (0.57)      & \multicolumn{1}{c}{\ci{0.10}{0.21}} &        \\
Parking: reserved garage (vs none) & 4.28^{***}   & -0.02       & 0.32 & 337.04 \\
                                   & (0.97)       & (0.32)      & \multicolumn{1}{c}{\ci{0.25}{0.40}} &        \\
Parking: reserved space (vs none)  & 3.90^{***}   & -4.64^{***} & 0.29                            & 306.98 \\
                                   & (0.82)       & (1.35)      & \multicolumn{1}{c}{\ci{0.23}{0.36}} &        \\
Price                              & -13.34^{***} &             &                                     &        \\
                                   & (2.61)       &             &                                     &        \\

\midrule
Num. obs.          & 1458    &             &        &        \\
Log Likelihood     & -612.41 &             &        &        \\
AIC                & 1314.82 &             &        &        \\
BIC                & 1552.64 &             &        &        \\
McFadden R\textsuperscript{2} & 0.39    &             &        &        \\
LR $\chi^{2}$ (df=9)          & 796.40  &             &        &        \\
p-value (LR)       & 0.00    &             &        &        \\
\bottomrule

\end{tabular}

\begin{tablenotes}
\scriptsize
\item \textit{Notes:} Estimation by maximum likelihood of the multinomial logit (MNL) and mixed logit (ML) models. 
Heteroskedasticity-robust standard errors are reported in parentheses. 
Marginal rate of substitution (MRS) and marginal willingness to pay (MWTP) estimates are computed from the ML model following the procedures described in the text. 
Ninety-five percent confidence intervals for MRS estimates are obtained using the delta method. 
$^{*}p < 0.1$; $^{**}p < 0.05$; $^{***}p < 0.01$.
\end{tablenotes}

\end{threeparttable}
\end{scriptsize}
\label{tab:base_renter}
\end{center}
\end{table}


Turning to renters, Table \ref{tab:base_renter} reveals similar preference patterns but with generally lower MWTP magnitudes. Renters are willing to pay about 618 SEK per month to reside within 500 m of green space and roughly 637 SEK for proximity to shops within 500 m—values that remain economically meaningful yet modestly below those of owners. Renters also show strong preferences for parking access, although their MWTP for a reserved garage (337 SEK) or space (307 SEK) is roughly half that of homeowners.

Comparing across tenure types, both groups place high value on proximity to green space and retail services, but the intensity of preferences differs. Owners systematically exhibit higher MWTP for most amenities, consistent with longer-term housing commitments and investment motivations. The larger heterogeneity in standard deviation estimates among renters suggests greater variation in preferences within this group, possibly reflecting shorter planning horizons or differing residential constraints.

Overall, these baseline results demonstrate robust and theoretically consistent patterns of residential preferences. The strong fit of the mixed logit models, along with significant random parameter estimates, supports their use as the preferred framework for the heterogeneity analyses that follow.




\clearpage


\subsection{Heterogenity Models}

Thus far our result point towards the presence of heterogeneous effects for housing preferences in regard to housing tenure. 
We next expand our testing by introducing cross-product terms in our mixed logit models framework to identify sources of preference heterogeneity in our sample.

\subsubsection{Age effects}

To investigate whether preferences vary systematically across the ageing process, we interacted all attribute levels with three age categories (55–64, 65–74, and 75+ years). The youngest group (55–64) serves as the reference category, and therefore the main effects represent their preferences. The interaction terms capture differences in preferences for the older groups relative to this baseline.

Table \ref{tab:age_group} shows that respondents aged 55–64 exhibit strong positive utilities and high MWTP values across all attributes. Among owners, the 55–64 group was willing to pay around 490 SEK/month to live within 500 m of shops, 429 SEK/month for green space within 500 m, and 453 SEK/month for a reserved garage. Among renters, corresponding values were higher, approximately 545 SEK, 498 SEK, and 323 SEK, respectively.

For the 65–74 group, MWTP values remain largely comparable to those of the 55–64 reference group. Differences are small, suggesting stability in preferences across early old age.
By contrast, respondents aged 75 and above display notable shifts in emphasis.
Among homeowners, MWTP for proximity to green areas and shops declines modestly to roughly 410 SEK and 240 SEK, while the value of parking amenities increases substantially, 
up to 700 SEK/month for a reserved garage and 640 SEK/month for a reserved space.

Among renters, MWTP in the oldest age group remains high for proximity and comfort attributes: approximately 580 SEK/month for nearby green areas, 400 SEK for shops, and 390–400 SEK for reserved parking.
Figure \ref{fig:wtp} illustrates MWTP values for across the age groups.

Results suggest that while all age groups value proximity and amenities, the relative importance of these attributes evolves across the ageing spectrum. Younger and mid-age respondents emphasize spatial accessibility,
being near shops, services, and green areas,
whereas the oldest group increasingly private parking and local environmental quality.


\begin{table}[H]
\caption{Hetergeneity models: Age groups}
\begin{center}
\begin{tiny}
\begin{threeparttable}
\begin{tabular}{l D{.}{.}{5.5} D{.}{.}{5.5} D{.}{.}{4.5} D{.}{.}{5.5}}
\toprule
 & \multicolumn{2}{c}{Owners} & \multicolumn{2}{c}{Renters} \\
\cmidrule(lr){2-3} \cmidrule(lr){4-5}
 & \multicolumn{1}{c}{Coef.} & \multicolumn{1}{c}{MWTP} & \multicolumn{1}{c}{Coef.} & \multicolumn{1}{c}{MWTP} \\
\midrule
Green space: 5 km (vs 15 km)                   & 1.34^{***}  & 207.89^{***}   & 4.56^{**}    & 267.46^{*}    \\
                                               & (0.24)      & (41.42)        & (1.50)       & (115.84)      \\
Green space: 500 m (vs 15 km)                  & 2.77^{***}  & 428.67^{***}   & 8.48^{***}   & 497.75^{**}   \\
                                               & (0.31)      & (59.48)        & (2.11)       & (166.14)      \\
Shops: 5 km (vs 15 km)                         & 1.24^{***}  & 192.54^{***}   & 4.44^{**}    & 260.62^{*}    \\
                                               & (0.28)      & (51.64)        & (1.62)       & (114.51)      \\
Shops: 500 m (vs 15 km)                        & 3.16^{***}  & 490.07^{***}   & 9.29^{***}   & 545.39^{***}  \\
                                               & (0.30)      & (62.12)        & (1.94)       & (119.10)      \\
Transit stop: 600 m (vs 900 m)                 & -0.06       & -9.81          & 2.61^{*}     & 153.07^{*}    \\
                                               & (0.24)      & (36.17)        & (1.04)       & (67.29)       \\
Transit stop: 300 m (vs 900 m)                 & 1.13^{***}  & 175.41^{***}   & 3.83^{***}   & 224.71^{**}   \\
                                               & (0.26)      & (42.69)        & (1.08)       & (73.92)       \\
Parking: reserved garage (vs none)             & 2.93^{***}  & 453.25^{***}   & 4.51^{***}   & 264.94^{***}  \\
                                               & (0.32)      & (62.52)        & (1.26)       & (60.67)       \\
Parking: reserved space (vs none)              & 2.17^{***}  & 336.81^{***}   & 5.49^{***}   & 322.26^{***}  \\
                                               & (0.29)      & (45.36)        & (1.29)       & (64.76)       \\
Price                                          & -6.78^{***} &                & -17.89^{***} &               \\
                                               & (0.71)      &                & (4.53)       &               \\
Green space: 5 km (vs 15 km) × Age 65–74       & -0.08       & 237.96^{***}   & -1.06        & 256.82^{***}  \\
                                               & (0.26)      & (35.32)        & (1.28)       & (58.54)       \\
Green space: 500 m (vs 15 km) × Age 65–74      & -0.47       & 432.51^{***}   & -2.66        & 427.55^{***}  \\
                                               & (0.34)      & (52.38)        & (1.58)       & (70.44)       \\
Shops: 5 km (vs 15 km) × Age 65–74             & -0.34       & 169.27^{***}   & -1.01        & 252.39^{***}  \\
                                               & (0.34)      & (44.61)        & (1.43)       & (67.91)       \\
Shops: 500 m (vs 15 km) × Age 65–74            & -0.33       & 533.49^{***}   & -1.71        & 556.94^{***}  \\
                                               & (0.31)      & (54.13)        & (1.35)       & (74.88)       \\
Transit stop: 600 m (vs 900 m) × Age 65–74     & 0.27        & 37.93          & -1.98        & 46.54         \\
                                               & (0.27)      & (30.21)        & (1.16)       & (42.94)       \\
Transit stop: 300 m (vs 900 m) × Age 65–74     & -0.02       & 208.46^{***}   & -2.90^{*}    & 68.64         \\
                                               & (0.31)      & (37.30)        & (1.14)       & (35.73)       \\
Parking: reserved garage (vs none) × Age 65–74 & 0.35        & 616.50^{***}   & -1.44        & 226.28^{***}  \\
                                               & (0.36)      & (60.90)        & (1.28)       & (45.21)       \\
Parking: reserved space (vs none) × Age 65–74  & 0.44        & 490.85^{***}   & -2.20        & 242.12^{***}  \\
                                               & (0.33)      & (47.56)        & (1.34)       & (45.88)       \\
Price × Age 65–74                              & 1.19        &                & 3.60         &   \\
                                               & (0.80)      & (182.90)       & (4.18)       & (388.70)      \\
Green space: 5 km (vs 15 km) × Age 75+         & -0.38       & 226.54^{***}   & 0.33         & 477.80^{***}  \\
                                               & (0.27)      & (45.28)        & (1.22)       & (101.36)      \\
Green space: 500 m (vs 15 km) × Age 75+        & -1.02^{**}  & 411.97^{***}   & -2.54        & 580.64^{***}  \\
                                               & (0.34)      & (66.71)        & (1.72)       & (127.64)      \\
Shops: 5 km (vs 15 km) × Age 75+               & -0.21       & 242.51^{***}   & -1.39        & 298.16^{**}   \\
                                               & (0.32)      & (60.27)        & (1.45)       & (96.09)       \\
Shops: 500 m (vs 15 km) × Age 75+              & -0.18       & 703.83^{***}   & -5.18^{**}   & 402.18^{***}  \\
                                               & (0.32)      & (88.20)        & (1.62)       & (99.04)       \\
Transit stop: 600 m (vs 900 m) × Age 75+       & 0.66^{*}    & 139.76^{**}    & -2.46        & 14.04         \\
                                               & (0.29)      & (46.13)        & (1.30)       & (68.86)       \\
Transit stop: 300 m (vs 900 m) × Age 75+       & 0.02        & 270.91^{***}   & -1.80        & 198.01^{**}   \\
                                               & (0.30)      & (54.04)        & (1.15)       & (70.29)       \\
Parking: reserved garage (vs none) × Age 75+   & 0.08        & 708.28^{***}   & -0.39        & 402.94^{***}  \\
                                               & (0.37)      & (86.08)        & (1.23)       & (86.56)       \\
Parking: reserved space (vs none) × Age 75+    & 0.55        & 642.81^{***}   & -1.47        & 392.44^{***}  \\
                                               & (0.34)      & (71.37)        & (1.25)       & (75.99)       \\
Price × Age 75+                                & 2.32^{**}   &                & 7.14         &         \\
                                               & (0.82)      & (229.50)       & (4.46)       & (504.58)      \\
\midrule
Num. obs.                                      & 7110        &                & 1458         &               \\
Log Likelihood                                 & -3119.74    &                & -604.05      &               \\
AIC                                            & 6365.47     &                & 1334.09      &               \\
BIC                                            & 6798.23     &                & 1667.04      &               \\
McFadden R²                                    & 0.37        &                & 0.40         &               \\
LR $\chi 2$ (df=63)                                  & 3617.08     &                & 813.13       &               \\
p-value (LR)                                   & 0.00        &                & 0.00         &               \\
\bottomrule
\end{tabular}

\begin{tablenotes}
\item \tiny{$^{***}p<0.001$; $^{**}p<0.01$; $^{*}p<0.05$}
\item \tiny{\textit{Note:} MWTP denotes marginal willingness to pay. Interaction terms indicate differential effects by age group.}
\end{tablenotes}
\end{threeparttable}
\end{tiny}
\label{tab:age_group}
\end{center}
\end{table}

\begin{table}[!htbp]
\caption{Interaction Effects — Median Age}
\begin{center}
\begin{scriptsize}
\begin{threeparttable}

\begin{tabular}{l D{.}{.}{5.5} c D{.}{.}{4.5} D{.}{.}{5.5} c D{.}{.}{5.2}}
\toprule
 & \multicolumn{3}{c}{Owner} & \multicolumn{3}{c}{Renter} \\
\cmidrule(lr){2-4} \cmidrule(lr){5-7}
 & \multicolumn{1}{c}{Coef.} & \multicolumn{1}{c}{MRS} & \multicolumn{1}{c}{MWTP} & \multicolumn{1}{c}{Coef.} & \multicolumn{1}{c}{MRS} & \multicolumn{1}{c}{MWTP} \\
\midrule

Green space: 5 km (vs 15 km)             & 1.08^{***}  & 0.18^{*}       & 165   & 2.22^{***}   & 0.19^{*}      & 194   \\
                                         & (0.15)      & \multicolumn{1}{c}{\ci{0.13}{0.24}} &          & (0.67)       & \multicolumn{1}{c}{\ci{0.09}{0.30}} &          \\
\addlinespace
Green space: 500 m (vs 15 km)            & 2.26^{***}  & 0.39^{*}       & 347   & 3.99^{***}   & 0.35^{*}      & 348   \\
                                         & (0.20)      & \multicolumn{1}{c}{\ci{0.31}{0.47}} &          & (0.86)       & \multicolumn{1}{c}{\ci{0.22}{0.47}} &          \\
\addlinespace
Shops: 5 km (vs 15 km)                   & 0.95^{***}  & 0.16^{*}       & 146   & 1.96^{**}    & 0.17^{*}      & 171   \\
                                         & (0.17)      & \multicolumn{1}{c}{\ci{0.09}{0.23}} &          & (0.67)       & \multicolumn{1}{c}{\ci{0.05}{0.29}} &          \\
\addlinespace
Shops: 500 m (vs 15 km)                  & 3.13^{***}  & 0.53^{*}       & 479   & 5.45^{***}   & 0.48^{*}      & 475   \\
                                         & (0.22)      & \multicolumn{1}{c}{\ci{0.44}{0.63}} &          & (0.98)       & \multicolumn{1}{c}{\ci{0.35}{0.61}} &          \\
\addlinespace
Transit stop: 600 m (vs 900 m)           & 0.09        & 0.02           & 14    & 0.72         & 0.06          & 63     \\
                                         & (0.14)      & \multicolumn{1}{c}{\ci{-0.03}{0.06}} &         & (0.42)       & \multicolumn{1}{c}{\ci{-0.01}{0.14}} &         \\
\addlinespace
Transit stop: 300 m (vs 900 m)           & 1.04^{***}  & 0.18^{*}       & 159   & 1.53^{***}   & 0.13^{*}      & 133   \\
                                         & (0.16)      & \multicolumn{1}{c}{\ci{0.12}{0.24}} &          & (0.43)       & \multicolumn{1}{c}{\ci{0.06}{0.21}} &          \\
\addlinespace
Parking: reserved garage (vs none)       & 3.02^{***}  & 0.51^{*}       & 463   & 2.79^{***}   & 0.24^{*}      & 243   \\
                                         & (0.23)      & \multicolumn{1}{c}{\ci{0.42}{0.61}} &          & (0.58)       & \multicolumn{1}{c}{\ci{0.16}{0.33}} &          \\
\addlinespace
Parking: reserved space (vs none)        & 2.24^{***}  & 0.38^{*}       & 343   & 3.02^{***}   & 0.26^{*}      & 263   \\
                                         & (0.20)      & \multicolumn{1}{c}{\ci{0.31}{0.45}} &          & (0.67)       & \multicolumn{1}{c}{\ci{0.17}{0.35}} &          \\
\addlinespace
Price                                    & -5.87^{***} &                &       & -11.46^{***} &               &        \\
                                         & (0.49)      &                &       & (2.16)       &               &        \\
\addlinespace
Green space: 5 km (vs 15 km) × Age       & -0.16       & 0.21^{*}       & 190   & 0.24         & 0.27^{*}      & 267   \\
                                         & (0.19)      & \multicolumn{1}{c}{\ci{0.14}{0.28}} &          & (0.77)       & \multicolumn{1}{c}{\ci{0.12}{0.41}} &          \\
\addlinespace
Green space: 500 m (vs 15 km) × Age      & -0.57^{*}   & 0.39^{*}       & 350   & -0.23        & 0.41^{*}      & 408   \\
                                         & (0.24)      & \multicolumn{1}{c}{\ci{0.29}{0.49}} &          & (0.83)       & \multicolumn{1}{c}{\ci{0.24}{0.58}} &          \\
\addlinespace
Shops: 5 km (vs 15 km) × Age             & -0.01       & 0.22^{*}       & 194   & -0.85        & 0.12          & 120   \\
                                         & (0.22)      & \multicolumn{1}{c}{\ci{0.12}{0.31}} &          & (0.80)       & \multicolumn{1}{c}{\ci{-0.02}{0.26}} &         \\
\addlinespace
Shops: 500 m (vs 15 km) × Age            & -0.23       & 0.66^{*}       & 597   & -1.94^{*}    & 0.38^{*}      & 381   \\
                                         & (0.24)      & \multicolumn{1}{c}{\ci{0.53}{0.80}} &          & (0.82)       & \multicolumn{1}{c}{\ci{0.25}{0.52}} &          \\
\addlinespace
Transit stop: 600 m (vs 900 m) × Age     & 0.50^{**}   & 0.14^{*}       & 123   & -0.49        & 0.03          & 26    \\
                                         & (0.20)      & \multicolumn{1}{c}{\ci{0.06}{0.21}} &          & (0.58)       & \multicolumn{1}{c}{\ci{-0.07}{0.12}} &         \\
\addlinespace
Transit stop: 300 m (vs 900 m) × Age     & 0.10        & 0.26^{*}       & 234   & -0.00        & 0.17^{*}      & 165   \\
                                         & (0.21)      & \multicolumn{1}{c}{\ci{0.18}{0.34}} &          & (0.56)       & \multicolumn{1}{c}{\ci{0.07}{0.26}} &          \\
\addlinespace
Parking: reserved garage (vs none) × Age & -0.10       & 0.67^{*}       & 604   & -0.51        & 0.25^{*}      & 248   \\
                                         & (0.27)      & \multicolumn{1}{c}{\ci{0.53}{0.81}} &          & (0.64)       & \multicolumn{1}{c}{\ci{0.14}{0.35}} &          \\
\addlinespace
Parking: reserved space (vs none) × Age  & 0.28        & 0.58^{*}       & 519   & -0.31        & 0.29^{*}      & 294   \\
                                         & (0.25)      & \multicolumn{1}{c}{\ci{0.47}{0.69}} &          & (0.73)       & \multicolumn{1}{c}{\ci{0.18}{0.40}} &          \\
\addlinespace
Price × Age                              & 1.51^{**}   &      &       & 2.26         &      &        \\
                                         & (0.58)      &  &         & (1.91)       &  &         \\
\addlinespace

\midrule
Num. obs.                                & 7110        &                &       & 1458         &               &        \\
Log Likelihood                           & -3143.01    &                &       & -623.13      &               &        \\
AIC                                      & 6394.01     &                &       & 1354.26      &               &        \\
BIC                                      & 6764.95     &                &       & 1639.64      &               &        \\
McFadden R\textsuperscript{2}           & 0.36        &                &       & 0.38         &               &        \\
LR $\chi^{2}$ (df=54)                   & 3570.54^{***}     &          &       & 774.96^{***}       &          &        \\

\bottomrule
\end{tabular}

\begin{tablenotes}
\scriptsize
\item \textit{Notes:} Estimation by maximum likelihood of the mixed logit model. Heteroskedasticity-robust standard errors are in parentheses. MRS and MWTP are calculated as described in the text. Confidence intervals for MRS are computed using the delta method. "Age" is a dummy equal to 1 if above the median. $^{*}p < 0.05$; $^{**}p < 0.01$; $^{***}p < 0.001$.
\end{tablenotes}

\end{threeparttable}
\end{scriptsize}
\label{table:interaction_median_age}
\end{center}
\end{table}






\begin{figure}[H]
\centering
\includegraphics[scale=0.30]{figures/wtp_age_interactions_own_rent_publication.png}
\caption{WTP estimates from model 2 \label{fig:wtp}}
\end{figure}


\clearpage


\subsubsection{Health}


\begin{table}
\caption{Mixed Logit with Health Interactions (Good vs Not): Coefficients, MRS (95% CI), and MWTP}
\begin{center}
\begin{scriptsize}
\begin{tabular}{l D{.}{.}{5.5} D{.}{.}{3.9} D{.}{.}{5.2} D{.}{.}{4.5} D{.}{.}{3.9} D{.}{.}{4.2}}
\toprule
 & \multicolumn{3}{c}{Owner} & \multicolumn{3}{c}{Renter} \\
\cmidrule(lr){2-4} \cmidrule(lr){5-7}
 & \multicolumn{1}{c}{Coef.} & \multicolumn{1}{c}{MRS} & \multicolumn{1}{c}{MWTP} & \multicolumn{1}{c}{Coef.} & \multicolumn{1}{c}{MRS} & \multicolumn{1}{c}{MWTP} \\
\midrule
Green space: 5 km (vs 15 km)              & 0.73^{**}   & 0.15^{*}       & 146.68   & 1.24        & 0.13           & 115.64  \\
                                          & (0.23)      & [ 0.05;  0.25] &          & (0.81)      & [-0.04;  0.30] &         \\
Green space: 500 m (vs 15 km)             & 1.47^{***}  & 0.29^{*}       & 292.75   & 2.69^{**}   & 0.28^{*}       & 251.69  \\
                                          & (0.29)      & [ 0.15;  0.43] &          & (1.01)      & [ 0.07;  0.49] &         \\
Shops: 5 km (vs 15 km)                    & 1.10^{***}  & 0.22^{*}       & 220.13   & 1.71        & 0.18           & 160.01  \\
                                          & (0.27)      & [ 0.08;  0.36] &          & (1.04)      & [-0.05;  0.41] &         \\
Shops: 500 m (vs 15 km)                   & 2.74^{***}  & 0.55^{*}       & 548.05   & 5.10^{***}  & 0.53^{*}       & 476.56  \\
                                          & (0.31)      & [ 0.37;  0.72] &          & (1.19)      & [ 0.28;  0.78] &         \\
Transit stop: 600 m (vs 900 m)            & 0.82^{***}  & 0.16^{*}       & 164.14   & 0.91        & 0.09           & 85.13   \\
                                          & (0.23)      & [ 0.05;  0.28] &          & (0.66)      & [-0.05;  0.24] &         \\
Transit stop: 300 m (vs 900 m)            & 1.46^{***}  & 0.29^{*}       & 291.50   & 2.70^{***}  & 0.28^{*}       & 251.97  \\
                                          & (0.28)      & [ 0.16;  0.42] &          & (0.76)      & [ 0.13;  0.43] &         \\
Parking: reserved garage (vs none)        & 2.97^{***}  & 0.59^{*}       & 594.41   & 2.44^{**}   & 0.25^{*}       & 228.34  \\
                                          & (0.35)      & [ 0.40;  0.79] &          & (0.80)      & [ 0.08;  0.42] &         \\
Parking: reserved space (vs none)         & 2.57^{***}  & 0.51^{*}       & 512.79   & 2.75^{**}   & 0.28^{*}       & 256.48  \\
                                          & (0.33)      & [ 0.35;  0.67] &          & (0.95)      & [ 0.12;  0.45] &         \\
Price                                     & -5.00^{***} &                &          & -9.64^{***} &                &         \\
                                          & (0.74)      &                &          & (2.46)      &                &         \\
Green space: 5 km (vs 15 km) × Good       & 0.35        & 0.21^{*}       & 211.37   & 1.43        & 0.25^{*}       & 223.43  \\
                                          & (0.25)      & [ 0.16;  0.26] &          & (0.99)      & [ 0.14;  0.35] &         \\
Green space: 500 m (vs 15 km) × Good      & 0.67^{*}    & 0.42^{*}       & 416.12   & 1.60        & 0.40^{*}       & 359.62  \\
                                          & (0.31)      & [ 0.34;  0.49] &          & (1.15)      & [ 0.27;  0.53] &         \\
Shops: 5 km (vs 15 km) × Good             & -0.20       & 0.18^{*}       & 176.21   & 0.18        & 0.18^{*}       & 158.42  \\
                                          & (0.29)      & [ 0.11;  0.24] &          & (1.13)      & [ 0.06;  0.29] &         \\
Shops: 500 m (vs 15 km) × Good            & 0.28        & 0.59^{*}       & 590.28   & -0.48       & 0.43^{*}       & 386.53  \\
                                          & (0.31)      & [ 0.50;  0.68] &          & (0.98)      & [ 0.32;  0.54] &         \\
Transit stop: 600 m (vs 900 m) × Good     & -0.58^{*}   & 0.05^{*}       & 47.26    & -0.61       & 0.03           & 24.92   \\
                                          & (0.25)      & [ 0.00;  0.09] &          & (0.76)      & [-0.04;  0.10] &         \\
Transit stop: 300 m (vs 900 m) × Good     & -0.40       & 0.21^{*}       & 206.34   & -1.58^{*}   & 0.10^{*}       & 93.34   \\
                                          & (0.29)      & [ 0.15;  0.26] &          & (0.79)      & [ 0.04;  0.17] &         \\
Parking: reserved garage (vs none) × Good & -0.01       & 0.58^{*}       & 579.21   & 0.13        & 0.24^{*}       & 215.53  \\
                                          & (0.36)      & [ 0.49;  0.67] &          & (0.82)      & [ 0.16;  0.31] &         \\
Parking: reserved space (vs none) × Good  & -0.19       & 0.46^{*}       & 464.21   & 0.24        & 0.28^{*}       & 249.64  \\
                                          & (0.34)      & [ 0.40;  0.53] &          & (0.96)      & [ 0.20;  0.36] &         \\
Price × Good                              & -0.12       & -1.00^{*}      & -1000.00 & -1.12       & -1.00^{*}      & -900.00 \\
                                          & (0.78)      & [-1.30; -0.70] &          & (2.47)      & [-1.76; -0.24] &         \\
\midrule
Num. obs.                                 & 7101        &                &          & 1440        &                &         \\
Log Likelihood                            & -3142.92    &                &          & -616.86     &                &         \\
AIC                                       & 6393.85     &                &          & 1341.71     &                &         \\
BIC                                       & 6764.72     &                &          & 1626.42     &                &         \\
McFadden R²                               & 0.36        &                &          & 0.38        &                &         \\
LR χ² (df=54)                             & 3558.23     &                &          & 762.55      &                &         \\
p-value (LR)                              & 0.00        &                &          & 0.00        &                &         \\
\bottomrule
\multicolumn{7}{l}{\tiny{$^{***}p<0.001$; $^{**}p<0.01$; $^{*}p<0.05$ (or Null hypothesis value outside the confidence interval).}}
\end{tabular}
\end{scriptsize}
\label{table:coefficients}
\end{center}
\end{table}




\clearpage


\subsubsection{Gender effects}


\begin{table}
\caption{Mixed Logit with Income Interactions: Retired vs Not Retired (Coefficients and MWTP per Row)}
\begin{center}
\begin{scriptsize}
\begin{tabular}{l D{.}{.}{5.5} D{.}{.}{6.11} D{.}{.}{4.5} D{.}{.}{6.10}}
\toprule
 & \multicolumn{2}{c}{Owner} & \multicolumn{2}{c}{Renter} \\
\cmidrule(lr){2-3} \cmidrule(lr){4-5}
 & \multicolumn{1}{c}{Coef.} & \multicolumn{1}{c}{MWTP} & \multicolumn{1}{c}{Coef.} & \multicolumn{1}{c}{MWTP} \\
\midrule
Green space: 5 km (vs 15 km)              & 0.91^{***}  & 188.30^{*}          & 2.34^{*}     & 187.03^{*}         \\
                                          & (0.24)      & (   76.03,  300.57) & (1.06)       & (   32.84, 341.21) \\
Green space: 500 m (vs 15 km)             & 1.57^{***}  & 323.43^{*}          & 4.76^{**}    & 380.78^{*}         \\
                                          & (0.29)      & (  174.72,  472.14) & (1.70)       & (  154.17, 607.39) \\
Shops: 5 km (vs 15 km)                    & 1.30^{***}  & 268.57^{*}          & 2.78^{*}     & 222.14^{*}         \\
                                          & (0.28)      & (  114.70,  422.45) & (1.31)       & (    8.72, 435.56) \\
Shops: 500 m (vs 15 km)                   & 2.78^{***}  & 571.28^{*}          & 7.91^{***}   & 632.61^{*}         \\
                                          & (0.31)      & (  384.84,  757.73) & (1.98)       & (  380.17, 885.04) \\
Transit stop: 600 m (vs 900 m)            & 0.86^{***}  & 177.57^{*}          & 1.39         & 111.30             \\
                                          & (0.24)      & (   60.23,  294.92) & (0.91)       & (  -30.01, 252.62) \\
Transit stop: 300 m (vs 900 m)            & 1.31^{***}  & 270.45^{*}          & 2.63^{*}     & 210.30^{*}         \\
                                          & (0.27)      & (  138.47,  402.43) & (1.03)       & ( 65.88, 354.73) \\
Parking: reserved garage (vs none)        & 3.27^{***}  & 673.15^{*}          & 4.70^{***}   & 375.71^{*}         \\
                                          & (0.36)      & (  456.60,  889.69) & (1.30)       & (  209.04, 542.37) \\
Parking: reserved space (vs none)         & 2.77^{***}  & 569.61^{*}          & 4.23^{**}    & 338.54^{*}         \\
                                          & (0.34)      & (  396.26,  742.96) & (1.35)       & (  185.23, 491.86) \\
Price                                     & -5.10^{***} &                     & -13.12^{***} &                    \\
                                          & (0.74)      &                     & (2.90)       &                    \\
Green space: 5 km (vs 15 km) × Good       & 0.33        & 243.79^{*}          & 2.78^{*}     & 345.98^{*}         \\
                                          & (0.25)      & (  190.27,  297.31) & (1.11)       & (252.80, 439.16) \\
Green space: 500 m (vs 15 km) × Good      & 0.70^{*}    & 445.92^{*}          & 2.80^{*}     & 511.20^{*}         \\
                                          & (0.30)      & (  372.00,  519.84) & (1.25)       & (393.88, 628.52) \\
Shops: 5 km (vs 15 km) × Good             & -0.25       & 206.70^{*}          & 0.31         & 208.46^{*}         \\
                                          & (0.30)      & (  140.77,  272.63) & (1.27)       & (99.47, 317.44) \\
Shops: 500 m (vs 15 km) × Good            & 0.51        & 643.80^{*}          & -1.15        & 457.10^{*}         \\
                                          & (0.31)      & (  548.57,  739.02) & (1.08)       & (349.69, 564.51) \\
Transit stop: 600 m (vs 900 m) × Good     & -0.58^{*}   & 54.92^{*}           & -1.30        & 6.18               \\
                                          & (0.26)      & (    8.63,  101.21) & (1.14)       & (-60.15,  72.52) \\
Transit stop: 300 m (vs 900 m) × Good     & -0.26       & 207.08^{*}          & -1.34        & 87.45^{*}          \\
                                          & (0.29)      & (  155.63,  258.53) & (1.08)       & (24.95, 149.96) \\
Parking: reserved garage (vs none) × Good & -0.27       & 587.34^{*}          & -0.85        & 259.98^{*}         \\
                                          & (0.36)      & (  500.74,  673.95) & (0.93)       & (  185.30, 334.66) \\
Parking: reserved space (vs none) × Good  & -0.35       & 473.43^{*}          & -0.58        & 246.70^{*}         \\
                                          & (0.34)      & (  406.79,  540.07) & (1.16)       & (180.22, 313.17) \\
Price × Good                              & -0.25       & -1050.00^{*}        & -2.40        & -1050.00           \\
                                          & (0.79)      & (-1345.23, -754.77) & (3.36)       & (-2106.55,   6.55) \\
\midrule
Num. obs.                                 & 7101        &                     & 1440         &                    \\
Log Likelihood                            & -3123.35    &                     & -600.89      &                    \\
AIC                                       & 6354.70     &                     & 1309.79      &                    \\
BIC                                       & 6725.57     &                     & 1594.50      &                    \\
McFadden R²                               & 0.37        &                     & 0.40         &                    \\
LR χ² (df=54)                             & 3597.38     &                     & 794.47       &                    \\
p-value (LR)                              & 0.00        &                     & 0.00         &                    \\
\bottomrule
\multicolumn{5}{l}{\tiny{$^{***}p<0.001$; $^{**}p<0.01$; $^{*}p<0.05$ (or Null hypothesis value outside the confidence interval).}}
\end{tabular}
\end{scriptsize}
\label{table:coefficients}
\end{center}
\end{table}


\begin{table}[!htbp]
\caption{Interaction Effects — Sex}
\begin{center}
\begin{scriptsize}
\begin{threeparttable}

\newcommand{\ci}[2]{(#1;\,#2)}

\begin{tabular}{l D{.}{.}{5.5} c D{.}{.}{5.2} D{.}{.}{5.5} c D{.}{.}{5.2}}
\toprule
 & \multicolumn{3}{c}{Owner} & \multicolumn{3}{c}{Renter} \\
\cmidrule(lr){2-4} \cmidrule(lr){5-7}
 & \multicolumn{1}{c}{Coef.} & \multicolumn{1}{c}{MRS} & \multicolumn{1}{c}{MWTP} & \multicolumn{1}{c}{Coef.} & \multicolumn{1}{c}{MRS} & \multicolumn{1}{c}{MWTP} \\
\midrule

Green space: 5 km (vs 15 km)             & 0.94^{***}  & 0.20^{*}       & 195   & 1.86^{**}    & 0.17^{*}       & 155   \\
                                         & (0.14)      & \multicolumn{1}{c}{\ci{0.13}{0.26}} &          & (0.59)       & \multicolumn{1}{c}{\ci{0.07}{0.27}} &         \\
\addlinespace
Green space: 500 m (vs 15 km)            & 2.31^{***}  & 0.48^{*}       & 478   & 3.89^{***}   & 0.36^{*}       & 324   \\
                                         & (0.19)      & \multicolumn{1}{c}{\ci{0.38}{0.58}} &          & (0.84)       & \multicolumn{1}{c}{\ci{0.23}{0.49}} &         \\
\addlinespace
Shops: 5 km (vs 15 km)                   & 0.73^{***}  & 0.15^{*}       & 151   & 2.01^{**}    & 0.19^{*}       & 167   \\
                                         & (0.16)      & \multicolumn{1}{c}{\ci{0.07}{0.23}} &          & (0.66)       & \multicolumn{1}{c}{\ci{0.07}{0.30}} &         \\
\addlinespace
Shops: 500 m (vs 15 km)                  & 2.99^{***}  & 0.62^{*}       & 618   & 4.40^{***}   & 0.41^{*}       & 366   \\
                                         & (0.22)      & \multicolumn{1}{c}{\ci{0.50}{0.73}} &          & (0.87)       & \multicolumn{1}{c}{\ci{0.30}{0.51}} &         \\
\addlinespace
Transit stop: 600 m (vs 900 m)           & 0.49^{***}  & 0.10^{*}       & 100   & 0.72^{*}     & 0.07^{*}       & 60    \\
                                         & (0.14)      & \multicolumn{1}{c}{\ci{0.04}{0.16}} &          & (0.35)       & \multicolumn{1}{c}{\ci{0.00}{0.13}} &         \\
\addlinespace
Transit stop: 300 m (vs 900 m)           & 1.39^{***}  & 0.29^{*}       & 288   & 1.95^{***}   & 0.18^{*}       & 162   \\
                                         & (0.16)      & \multicolumn{1}{c}{\ci{0.21}{0.36}} &          & (0.47)       & \multicolumn{1}{c}{\ci{0.10}{0.26}} &         \\
\addlinespace
Parking: reserved garage (vs none)       & 2.66^{***}  & 0.55^{*}       & 550   & 2.72^{***}   & 0.25^{*}       & 226   \\
                                         & (0.22)      & \multicolumn{1}{c}{\ci{0.44}{0.66}} &          & (0.58)       & \multicolumn{1}{c}{\ci{0.16}{0.34}} &         \\
\addlinespace
Parking: reserved space (vs none)        & 2.08^{***}  & 0.43^{*}       & 430   & 2.93^{***}   & 0.27^{*}       & 244   \\
                                         & (0.19)      & \multicolumn{1}{c}{\ci{0.35}{0.51}} &          & (0.65)       & \multicolumn{1}{c}{\ci{0.19}{0.35}} &         \\
\addlinespace
Price                                    & -4.83^{***} &                &       & -10.81^{***} &                &         \\
                                         & (0.43)      &                &       & (2.02)       &                &         \\
\addlinespace
Green space: 5 km (vs 15 km) × Men       & 0.19        & 0.21^{*}       & 207   & 1.08         & 0.33^{*}       & 300   \\
                                         & (0.19)      & \multicolumn{1}{c}{\ci{0.14}{0.27}} &          & (0.81)       & \multicolumn{1}{c}{\ci{0.15}{0.52}} &         \\
\addlinespace
Green space: 500 m (vs 15 km) × Men      & -0.59^{*}   & 0.31^{*}       & 313   & -0.15        & 0.42^{*}       & 381   \\
                                         & (0.23)      & \multicolumn{1}{c}{\ci{0.23}{0.39}} &          & (0.85)       & \multicolumn{1}{c}{\ci{0.21}{0.64}} &         \\
\addlinespace
Shops: 5 km (vs 15 km) × Men             & 0.45^{*}    & 0.22^{*}       & 215   & -0.28        & 0.20           & 176   \\
                                         & (0.22)      & \multicolumn{1}{c}{\ci{0.13}{0.30}} &          & (0.85)       & \multicolumn{1}{c}{\ci{-0.00}{0.39}} &         \\
\addlinespace
Shops: 500 m (vs 15 km) × Men            & 0.19        & 0.58^{*}       & 578   & 0.60         & 0.57^{*}       & 510   \\
                                         & (0.24)      & \multicolumn{1}{c}{\ci{0.47}{0.69}} &          & (0.79)       & \multicolumn{1}{c}{\ci{0.34}{0.80}} &         \\
\addlinespace
Transit stop: 600 m (vs 900 m) × Men     & -0.24       & 0.04           & 44    & -0.44        & 0.03           & 29    \\
                                         & (0.20)      & \multicolumn{1}{c}{\ci{-0.01}{0.10}} &         & (0.63)       & \multicolumn{1}{c}{\ci{-0.09}{0.15}} &         \\
\addlinespace
Transit stop: 300 m (vs 900 m) × Men     & -0.53^{*}   & 0.16^{*}       & 157   & -1.07        & 0.10           & 89    \\
                                         & (0.22)      & \multicolumn{1}{c}{\ci{0.09}{0.22}} &          & (0.65)       & \multicolumn{1}{c}{\ci{-0.01}{0.21}} &         \\
\addlinespace
Parking: reserved garage (vs none) × Men & 0.77^{**}   & 0.62^{*}       & 624   & -0.08        & 0.30^{*}       & 268   \\
                                         & (0.27)      & \multicolumn{1}{c}{\ci{0.51}{0.74}} &          & (0.71)       & \multicolumn{1}{c}{\ci{0.14}{0.45}} &         \\
\addlinespace
Parking: reserved space (vs none) × Men  & 0.77^{**}   & 0.52^{*}       & 518   & -0.49        & 0.28^{*}       & 249   \\
                                         & (0.25)      & \multicolumn{1}{c}{\ci{0.43}{0.61}} &          & (0.77)       & \multicolumn{1}{c}{\ci{0.14}{0.41}} &         \\
\addlinespace
Price × Men                              & -0.66       &                &       & 1.99         &                &       \\
                                         & (0.58)      &  &         & (1.95)       &  &         \\

\midrule
Num. obs.                                & 7110        &                &       & 1458         &                &         \\
Log Likelihood                           & -3140.82    &                &       & -620.70      &                &         \\
AIC                                      & 6389.63     &                &       & 1349.40      &                &         \\
BIC                                      & 6760.57     &                &       & 1634.78      &                &         \\
McFadden R\textsuperscript{2}           & 0.36        &                &       & 0.39         &                &         \\
LR $\chi^{2}$ (df=54)                   & 3574.92^{***}     &          &       & 779.82^{***}       &          &         \\

\bottomrule
\end{tabular}

\begin{tablenotes}
\scriptsize
\item \textit{Notes:} Estimation by maximum likelihood of the mixed logit model. Heteroskedasticity-robust standard errors are in parentheses. MRS and MWTP are calculated as described in the text. Confidence intervals for MRS are computed using the delta method. "Men" is a dummy equal to 1 if the respondent is male. $^{*}p < 0.05$; $^{**}p < 0.01$; $^{***}p < 0.001$.
\end{tablenotes}

\end{threeparttable}
\end{scriptsize}
\label{table:interaction_sex}
\end{center}
\end{table}



\section{Discussion}

This study explored preference heterogeneity among older individuals considering relocation in Sweden.
To our knowledge, 
this is the first study to utilize a discrete choice experiment to examine locational preferences in Sweden.

\section{Limitations}



\section{Conclusions}

As the global population continues to age, understanding the housing preferences of older demographics becomes increasingly crucial for the planning and development of future societies.
In Sweden, where a significant portion of older individuals prefer to live in their own homes, the need for appropriate housing options for this segment is becoming more pronounced as this demographic group expands. 

This study utilized a discrete choice experiment to delve into factors influencing the housing choices of older individuals in Sweden considering relocation.
By presenting hypothetical scenarios that varied in locational attributes such as healthcare facilities, public transportation, green spaces, social amenities, and natural surroundings, we were able to estimate willingness-to-pay values and compare differences across various groups.

%We find that respondents in older age groups demonstrate differences in preferred housing attributes.
%Individual aged 75+ are willing to pay more to be closer to public transpiration compared to individuals aged 55-64.
%
%Across all tests, access to parking and proximity to shops emerged as paramount factors influencing housing choices. These findings offer valuable insights for rural planners, policy makers, and healthcare providers, providing guidance on creating age-friendly environments that cater to the unique needs and desires of older home owners while promoting resilient and vibrant communities. Understanding the interplay between these key factors is essential for optimizing the living experience for the older population, ensuring their well-being, and fostering their continued contribution to the vitality of future societies. As populations continue to age, these insights will be instrumental in shaping the future of housing and community development for older adults.



\newpage
\clearpage

\section{Appendix}


\begin{table}[h]
\caption{Baseline results - Men}
\begin{center}
\begin{scriptsize}
\begin{tabular}{l D{.}{.}{5.5} D{.}{.}{5.5} D{.}{.}{2.5} D{.}{.}{2.5} D{.}{.}{3.2}}
\toprule
 & & \multicolumn{2}{c}{ML} \\
\cmidrule(lr){3-4}
 & \multicolumn{1}{c}{MNL} & \multicolumn{1}{c}{Mean} & \multicolumn{1}{c}{SD} & \multicolumn{1}{c}{MRS} & \multicolumn{1}{c}{MWTP} \\
\midrule
Green space: 5 km (vs 15 km)       & 0.65^{***}  & 1.10^{***}  & -0.03      & 0.20^{***} & 206.44 \\
                                   & (0.07)      & (0.17)      & (0.32)     & (0.03)     &        \\
Green space: 500 m (vs 15 km)      & 1.01^{***}  & 1.78^{***}  & 0.20       & 0.32^{***} & 334.09 \\
                                   & (0.07)      & (0.21)      & (0.53)     & (0.04)     &        \\
Shops: 5 km (vs 15 km)             & 0.69^{***}  & 1.14^{***}  & 0.13       & 0.20^{***} & 213.22 \\
                                   & (0.08)      & (0.18)      & (0.33)     & (0.04)     &        \\
Shops: 500 m (vs 15 km)            & 1.76^{***}  & 3.46^{***}  & 0.76       & 0.62^{***} & 649.92 \\
                                   & (0.07)      & (0.29)      & (0.41)     & (0.06)     &        \\
Transit stop: 600 m (vs 900 m)     & 0.20^{**}   & 0.33^{*}    & 0.01       & 0.06^{*}   & 62.86  \\
                                   & (0.08)      & (0.15)      & (0.31)     & (0.03)     &        \\
Transit stop: 300 m (vs 900 m)     & 0.41^{***}  & 1.01^{***}  & 0.51^{*}   & 0.18^{***} & 188.91 \\
                                   & (0.07)      & (0.18)      & (0.24)     & (0.03)     &        \\
Parking: reserved garage (vs none) & 1.71^{***}  & 3.42^{***}  & 1.04^{***} & 0.61^{***} & 642.86 \\
                                   & (0.08)      & (0.28)      & (0.25)     & (0.06)     &        \\
Parking: reserved space (vs none)  & 1.43^{***}  & 2.85^{***}  & 2.43^{***} & 0.51^{***} & 534.90 \\
                                   & (0.08)      & (0.25)      & (0.25)     & (0.05)     &        \\
Price                              & -3.01^{***} & -5.59^{***} &            &            &        \\
                                   & (0.22)      & (0.49)      &            &            &        \\
\midrule
Num. obs.                          & 3852        & 3852        &            &            &        \\
Log Likelihood                     & -1889.12    & -1654.13    &            &            &        \\
AIC                                & 3796.25     & 3398.26     &            &            &        \\
BIC                                & 3852.55     & 3679.79     &            &            &        \\
McFadden R²                        & 0.29        & 0.38        &            &            &        \\
LR $\chi 2$ (df=9)                       & 1561.76     & 2031.75     &            &            &        \\
p-value (LR)                       & 0.00        & 0.00        &            &            &        \\
\bottomrule
\multicolumn{6}{l}{\tiny{$^{***}p<0.001$; $^{**}p<0.01$; $^{*}p<0.05$}}
\end{tabular}
\end{scriptsize}
\label{table:coefficients}
\end{center}
\end{table}


\begin{table}[h]
\caption{Baseline results - Female}
\label{table:base_male}
\begin{center}
\scriptsize
\begin{tabular}{l D{.}{.}{5.5} D{.}{.}{5.5} D{.}{.}{2.5} D{.}{.}{2.5} D{.}{.}{3.2}}
\toprule
 & & \multicolumn{2}{c}{MXL} \\
\cmidrule(lr){3-4}
 & \multicolumn{1}{c}{MNL} & \multicolumn{1}{c}{Mean} & \multicolumn{1}{c}{SD} & \multicolumn{1}{c}{MRS} & \multicolumn{1}{c}{MWTP (SEK/mo)} \\
\midrule
Green space: 5 km (vs 15 km)       & 0.56^{***}  & 1.25^{***}  & 0.25        & 0.23^{***} & 242.97 \\
                                   & (0.06)      & (0.17)      & (0.54)      & (0.03)     &        \\
Green space: 500 m (vs 15 km)      & 1.24^{***}  & 2.76^{***}  & 1.24^{**}   & 0.51^{***} & 537.22 \\
                                   & (0.06)      & (0.24)      & (0.43)      & (0.05)     &        \\
Shops: 5 km (vs 15 km)             & 0.55^{***}  & 1.07^{***}  & 0.86^{**}   & 0.20^{***} & 207.50 \\
                                   & (0.06)      & (0.21)      & (0.29)      & (0.04)     &        \\
Shops: 500 m (vs 15 km)            & 1.61^{***}  & 3.10^{***}  & 1.33^{***}  & 0.57^{***} & 602.77 \\
                                   & (0.06)      & (0.24)      & (0.39)      & (0.05)     &        \\
Transit stop: 600 m (vs 900 m)     & 0.24^{***}  & 0.50^{***}  & -0.09       & 0.09^{***} & 98.21  \\
                                   & (0.07)      & (0.14)      & (0.32)      & (0.03)     &        \\
Transit stop: 300 m (vs 900 m)     & 0.65^{***}  & 1.38^{***}  & 1.03^{***}  & 0.26^{***} & 268.90 \\
                                   & (0.06)      & (0.16)      & (0.22)      & (0.03)     &        \\
Parking: reserved garage (vs none) & 1.21^{***}  & 2.43^{***}  & 0.41        & 0.45^{***} & 473.25 \\
                                   & (0.07)      & (0.20)      & (0.26)      & (0.04)     &        \\
Parking: reserved space (vs none)  & 1.10^{***}  & 1.96^{***}  & -2.17^{***} & 0.36^{***} & 381.55 \\
                                   & (0.07)      & (0.18)      & (0.25)      & (0.03)     &        \\
Price                              & -3.07^{***} & -5.40^{***} &             &            &        \\
                                   & (0.19)      & (0.44)      &             &            &        \\
\midrule
Num. obs.                          & 4761        & 4761        &             &            &        \\
Log Likelihood                     & -2431.00    & -2122.98    &             &            &        \\
AIC                                & 4880.00     & 4335.96     &             &            &        \\
BIC                                & 4938.21     & 4627.03     &             &            &        \\
McFadden R²                        & 0.26        & 0.36        &             &            &        \\
LR $\chi 2$ (df=9)                       & 1738.15     & 2354.18     &             &            &        \\
p-value (LR)                       & 0.00        & 0.00        &             &            &        \\
\bottomrule
\multicolumn{6}{l}{\scriptsize{$^{***}p<0.001$; $^{**}p<0.01$; $^{*}p<0.05$}}
\end{tabular}
\end{center}
\end{table}



\begin{table}
\caption{Baseline results - 55-64}
\begin{center}
\begin{scriptsize}
\begin{tabular}{l D{.}{.}{4.5} D{.}{.}{4.5} D{.}{.}{2.5} D{.}{.}{2.5} D{.}{.}{3.2}}
\toprule
 & & \multicolumn{2}{c}{ML} \\
\cmidrule(lr){3-4}
 & \multicolumn{1}{c}{MNL} & \multicolumn{1}{c}{Mean} & \multicolumn{1}{c}{SD} & \multicolumn{1}{c}{MRS} & \multicolumn{1}{c}{MWTP} \\
\midrule
Green space: 5 km (vs 15 km)       & 0.63^{***}  & 1.95^{***}   & -0.63       & 0.19^{***} & 200.97 \\
                                   & (0.10)      & (0.40)       & (0.48)      & (0.03)     &        \\
Green space: 500 m (vs 15 km)      & 1.45^{***}  & 4.81^{***}   & 4.35^{***}  & 0.47^{***} & 495.74 \\
                                   & (0.11)      & (0.73)       & (0.82)      & (0.05)     &        \\
Shops: 5 km (vs 15 km)             & 0.65^{***}  & 1.67^{***}   & 0.40        & 0.16^{***} & 172.18 \\
                                   & (0.12)      & (0.41)       & (0.39)      & (0.04)     &        \\
Shops: 500 m (vs 15 km)            & 1.76^{***}  & 4.94^{***}   & 1.07^{*}    & 0.48^{***} & 509.16 \\
                                   & (0.11)      & (0.65)       & (0.44)      & (0.05)     &        \\
Transit stop: 600 m (vs 900 m)     & 0.10        & 0.59         & 1.90^{***}  & 0.06       & 60.54  \\
                                   & (0.11)      & (0.33)       & (0.44)      & (0.03)     &        \\
Transit stop: 300 m (vs 900 m)     & 0.60^{***}  & 1.89^{***}   & -0.08       & 0.19^{***} & 194.66 \\
                                   & (0.11)      & (0.38)       & (0.47)      & (0.03)     &        \\
Parking: reserved garage (vs none) & 1.31^{***}  & 3.84^{***}   & -1.66^{***} & 0.38^{***} & 395.29 \\
                                   & (0.11)      & (0.68)       & (0.45)      & (0.05)     &        \\
Parking: reserved space (vs none)  & 1.04^{***}  & 3.19^{***}   & 3.84^{***}  & 0.31^{***} & 328.33 \\
                                   & (0.12)      & (0.54)       & (0.70)      & (0.04)     &        \\
Price                              & -3.99^{***} & -10.19^{***} &             &            &        \\
                                   & (0.36)      & (1.42)       &             &            &        \\
\midrule
Num. obs.                          & 1728        & 1728         &             &            &        \\
Log Likelihood                     & -817.95     & -689.10      &             &            &        \\
AIC                                & 1653.91     & 1468.20      &             &            &        \\
BIC                                & 1703.00     & 1713.66      &             &            &        \\
McFadden R²                        & 0.32        & 0.42         &             &            &        \\
LR $\chi 2$ (df=9)                       & 759.61      & 1017.32      &             &            &        \\
p-value (LR)                       & 0.00        & 0.00         &             &            &        \\
\bottomrule
\multicolumn{6}{l}{\tiny{$^{***}p<0.001$; $^{**}p<0.01$; $^{*}p<0.05$}}
\end{tabular}
\end{scriptsize}
\label{table:55_64}
\end{center}
\end{table}


\begin{table}
\caption{Baseline results - 65-74}
\begin{center}
\begin{scriptsize}
\begin{tabular}{l D{.}{.}{5.5} D{.}{.}{5.5} D{.}{.}{2.5} D{.}{.}{2.5} D{.}{.}{3.2}}
\toprule
 & & \multicolumn{2}{c}{ML} \\
\cmidrule(lr){3-4}
 & \multicolumn{1}{c}{MNL} & \multicolumn{1}{c}{Mean} & \multicolumn{1}{c}{SD} & \multicolumn{1}{c}{MRS} & \multicolumn{1}{c}{MWTP} \\
\midrule
Green space: 5 km (vs 15 km)       & 0.62^{***}  & 1.38^{***}  & 0.38       & 0.20^{***} & 214.30 \\
                                   & (0.07)      & (0.22)      & (0.24)     & (0.03)     &        \\
Green space: 500 m (vs 15 km)      & 1.18^{***}  & 2.66^{***}  & 0.53       & 0.39^{***} & 411.97 \\
                                   & (0.07)      & (0.33)      & (0.40)     & (0.04)     &        \\
Shops: 5 km (vs 15 km)             & 0.54^{***}  & 1.05^{***}  & 0.52       & 0.16^{***} & 163.48 \\
                                   & (0.07)      & (0.24)      & (0.38)     & (0.04)     &        \\
Shops: 500 m (vs 15 km)            & 1.70^{***}  & 3.72^{***}  & 2.22^{***} & 0.55^{***} & 576.94 \\
                                   & (0.08)      & (0.39)      & (0.29)     & (0.05)     &        \\
Transit stop: 600 m (vs 900 m)     & 0.20^{**}   & 0.24        & -0.67^{**} & 0.04       & 36.99  \\
                                   & (0.08)      & (0.16)      & (0.24)     & (0.02)     &        \\
Transit stop: 300 m (vs 900 m)     & 0.54^{***}  & 1.10^{***}  & 1.19^{***} & 0.16^{***} & 170.35 \\
                                   & (0.07)      & (0.19)      & (0.25)     & (0.03)     &        \\
Parking: reserved garage (vs none) & 1.51^{***}  & 3.61^{***}  & 1.08^{***} & 0.53^{***} & 559.50 \\
                                   & (0.08)      & (0.37)      & (0.25)     & (0.05)     &        \\
Parking: reserved space (vs none)  & 1.26^{***}  & 2.69^{***}  & 2.55^{***} & 0.40^{***} & 416.65 \\
                                   & (0.08)      & (0.28)      & (0.38)     & (0.03)     &        \\
Price                              & -3.09^{***} & -6.77^{***} &            &            &        \\
                                   & (0.22)      & (0.65)      &            &            &        \\
\midrule
Num. obs.                          & 3744        & 3744        &            &            &        \\
Log Likelihood                     & -1868.23    & -1646.37    &            &            &        \\
AIC                                & 3754.45     & 3382.73     &            &            &        \\
BIC                                & 3810.50     & 3662.99     &            &            &        \\
McFadden R²                        & 0.28        & 0.37        &            &            &        \\
LR $\chi 2$ (df=9)                       & 1453.83     & 1897.56     &            &            &        \\
p-value (LR)                       & 0.00        & 0.00        &            &            &        \\
\bottomrule
\multicolumn{6}{l}{\tiny{$^{***}p<0.001$; $^{**}p<0.01$; $^{*}p<0.05$}}
\end{tabular}
\end{scriptsize}
\label{table:65_74}
\end{center}
\end{table}


\begin{table}
\caption{Mixed Logit Estimates for 75+ : Base Specification}
\begin{center}
\begin{scriptsize}
\begin{tabular}{l D{.}{.}{5.5} D{.}{.}{5.5} D{.}{.}{2.5} D{.}{.}{2.5} D{.}{.}{3.2}}
\toprule
 & & \multicolumn{2}{c}{MXL} \\
\cmidrule(lr){3-4}
 & \multicolumn{1}{c}{MNL} & \multicolumn{1}{c}{Mean} & \multicolumn{1}{c}{SD} & \multicolumn{1}{c}{MRS} & \multicolumn{1}{c}{MWTP (SEK/mo)} \\
\midrule
Green space: 5 km (vs 15 km)       & 0.55^{***}  & 1.10^{***}  & -0.17       & 0.23^{***} & 238.16 \\
                                   & (0.07)      & (0.21)      & (0.28)      & (0.05)     &        \\
Green space: 500 m (vs 15 km)      & 0.95^{***}  & 2.01^{***}  & 0.40        & 0.41^{***} & 434.70 \\
                                   & (0.08)      & (0.25)      & (0.36)      & (0.06)     &        \\
Shops: 5 km (vs 15 km)             & 0.66^{***}  & 1.21^{***}  & 0.47        & 0.25^{***} & 261.18 \\
                                   & (0.08)      & (0.22)      & (0.28)      & (0.06)     &        \\
Shops: 500 m (vs 15 km)            & 1.59^{***}  & 3.34^{***}  & 0.13        & 0.69^{***} & 722.79 \\
                                   & (0.08)      & (0.33)      & (0.25)      & (0.08)     &        \\
Transit stop: 600 m (vs 900 m)     & 0.32^{***}  & 0.55^{***}  & -0.70^{***} & 0.11^{**}  & 119.57 \\
                                   & (0.08)      & (0.16)      & (0.21)      & (0.04)     &        \\
Transit stop: 300 m (vs 900 m)     & 0.52^{***}  & 1.20^{***}  & 0.57^{*}    & 0.25^{***} & 258.91 \\
                                   & (0.08)      & (0.20)      & (0.25)      & (0.04)     &        \\
Parking: reserved garage (vs none) & 1.42^{***}  & 3.04^{***}  & 0.61^{*}    & 0.63^{***} & 658.01 \\
                                   & (0.08)      & (0.29)      & (0.24)      & (0.08)     &        \\
Parking: reserved space (vs none)  & 1.33^{***}  & 2.76^{***}  & -2.61^{***} & 0.57^{***} & 597.03 \\
                                   & (0.09)      & (0.26)      & (0.31)      & (0.06)     &        \\
Price                              & -2.51^{***} & -4.86^{***} &             &            &        \\
                                   & (0.23)      & (0.55)      &             &            &        \\
\midrule
Num. obs.                          & 3141        & 3141        &             &            &        \\
Log Likelihood                     & -1632.69    & -1425.97    &             &            &        \\
AIC                                & 3283.38     & 2941.93     &             &            &        \\
BIC                                & 3337.85     & 3214.29     &             &            &        \\
McFadden R²                        & 0.25        & 0.35        &             &            &        \\
LR χ² (df=9)                       & 1088.97     & 1502.42     &             &            &        \\
p-value (LR)                       & 0.00        & 0.00        &             &            &        \\
\bottomrule
\multicolumn{6}{l}{\tiny{$^{***}p<0.001$; $^{**}p<0.01$; $^{*}p<0.05$}}
\end{tabular}
\end{scriptsize}
\label{table:coefficients}
\end{center}
\end{table}


\begin{table}
\caption{Mixed Logit Estimates for retired : Base Specification}
\begin{center}
\begin{scriptsize}
\begin{tabular}{l D{.}{.}{5.5} D{.}{.}{5.5} D{.}{.}{2.5} D{.}{.}{2.5} D{.}{.}{3.2}}
\toprule
 & & \multicolumn{2}{c}{MXL} \\
\cmidrule(lr){3-4}
 & \multicolumn{1}{c}{MNL} & \multicolumn{1}{c}{Mean} & \multicolumn{1}{c}{SD} & \multicolumn{1}{c}{MRS} & \multicolumn{1}{c}{MWTP (SEK/mo)} \\
\midrule
Green space: 5 km (vs 15 km)       & 0.60^{***}  & 1.00^{***}  & 0.01       & 0.20^{***} & 208.42 \\
                                   & (0.05)      & (0.12)      & (0.23)     & (0.03)     &        \\
Green space: 500 m (vs 15 km)      & 1.10^{***}  & 1.86^{***}  & 0.27       & 0.37^{***} & 388.08 \\
                                   & (0.05)      & (0.15)      & (0.43)     & (0.03)     &        \\
Shops: 5 km (vs 15 km)             & 0.58^{***}  & 1.04^{***}  & 1.07^{***} & 0.21^{***} & 216.03 \\
                                   & (0.06)      & (0.15)      & (0.20)     & (0.03)     &        \\
Shops: 500 m (vs 15 km)            & 1.65^{***}  & 3.10^{***}  & 0.43       & 0.62^{***} & 646.36 \\
                                   & (0.06)      & (0.20)      & (0.29)     & (0.05)     &        \\
Transit stop: 600 m (vs 900 m)     & 0.26^{***}  & 0.49^{***}  & 1.00^{***} & 0.10^{***} & 103.05 \\
                                   & (0.06)      & (0.11)      & (0.19)     & (0.02)     &        \\
Transit stop: 300 m (vs 900 m)     & 0.55^{***}  & 1.05^{***}  & 0.18       & 0.21^{***} & 218.75 \\
                                   & (0.06)      & (0.12)      & (0.17)     & (0.03)     &        \\
Parking: reserved garage (vs none) & 1.50^{***}  & 2.92^{***}  & 1.11^{***} & 0.58^{***} & 609.32 \\
                                   & (0.06)      & (0.18)      & (0.20)     & (0.05)     &        \\
Parking: reserved space (vs none)  & 1.35^{***}  & 2.52^{***}  & 2.14^{***} & 0.50^{***} & 525.65 \\
                                   & (0.06)      & (0.19)      & (0.18)     & (0.04)     &        \\
Price                              & -2.88^{***} & -5.03^{***} &            &            &        \\
                                   & (0.16)      & (0.38)      &            &            &        \\
\midrule
Num. obs.                          & 6642        & 6642        &            &            &        \\
Log Likelihood                     & -3364.20    & -2978.20    &            &            &        \\
AIC                                & 6746.39     & 6046.40     &            &            &        \\
BIC                                & 6807.60     & 6352.46     &            &            &        \\
McFadden R²                        & 0.27        & 0.35        &            &            &        \\
LR χ² (df=9)                       & 2479.37     & 3251.36     &            &            &        \\
p-value (LR)                       & 0.00        & 0.00        &            &            &        \\
\bottomrule
\multicolumn{6}{l}{\tiny{$^{***}p<0.001$; $^{**}p<0.01$; $^{*}p<0.05$}}
\end{tabular}
\end{scriptsize}
\label{table:coefficients}
\end{center}
\end{table}


\begin{table}
\caption{Mixed Logit Estimates for notretired : Base Specification}
\begin{center}
\begin{scriptsize}
\begin{tabular}{l D{.}{.}{4.5} D{.}{.}{4.5} D{.}{.}{2.5} D{.}{.}{2.5} D{.}{.}{3.2}}
\toprule
 & & \multicolumn{2}{c}{MXL} \\
\cmidrule(lr){3-4}
 & \multicolumn{1}{c}{MNL} & \multicolumn{1}{c}{Mean} & \multicolumn{1}{c}{SD} & \multicolumn{1}{c}{MRS} & \multicolumn{1}{c}{MWTP (SEK/mo)} \\
\midrule
Green space: 5 km (vs 15 km)       & 0.58^{***}  & 1.40^{***}  & 0.52        & 0.16^{***} & 171.51 \\
                                   & (0.09)      & (0.27)      & (0.28)      & (0.03)     &        \\
Green space: 500 m (vs 15 km)      & 1.29^{***}  & 3.27^{***}  & -0.86^{**}  & 0.38^{***} & 399.24 \\
                                   & (0.10)      & (0.42)      & (0.29)      & (0.04)     &        \\
Shops: 5 km (vs 15 km)             & 0.69^{***}  & 1.63^{***}  & 0.65^{*}    & 0.19^{***} & 199.17 \\
                                   & (0.11)      & (0.36)      & (0.33)      & (0.04)     &        \\
Shops: 500 m (vs 15 km)            & 1.74^{***}  & 4.48^{***}  & 1.51^{***}  & 0.52^{***} & 546.79 \\
                                   & (0.10)      & (0.57)      & (0.36)      & (0.06)     &        \\
Transit stop: 600 m (vs 900 m)     & 0.11        & 0.37        & 0.26        & 0.04       & 44.95  \\
                                   & (0.11)      & (0.25)      & (0.43)      & (0.03)     &        \\
Transit stop: 300 m (vs 900 m)     & 0.54^{***}  & 1.61^{***}  & 1.39^{***}  & 0.19^{***} & 196.08 \\
                                   & (0.10)      & (0.27)      & (0.25)      & (0.03)     &        \\
Parking: reserved garage (vs none) & 1.20^{***}  & 3.16^{***}  & -1.60^{***} & 0.37^{***} & 385.74 \\
                                   & (0.10)      & (0.40)      & (0.33)      & (0.05)     &        \\
Parking: reserved space (vs none)  & 0.88^{***}  & 2.32^{***}  & 3.38^{***}  & 0.27^{***} & 283.63 \\
                                   & (0.11)      & (0.38)      & (0.51)      & (0.04)     &        \\
Price                              & -3.60^{***} & -8.60^{***} &             &            &        \\
                                   & (0.33)      & (1.10)      &             &            &        \\
\midrule
Num. obs.                          & 1935        & 1935        &             &            &        \\
Log Likelihood                     & -946.20     & -800.34     &             &            &        \\
AIC                                & 1910.40     & 1690.69     &             &            &        \\
BIC                                & 1960.51     & 1941.24     &             &            &        \\
McFadden R²                        & 0.29        & 0.40        &             &            &        \\
LR χ² (df=9)                       & 790.08      & 1081.79     &             &            &        \\
p-value (LR)                       & 0.00        & 0.00        &             &            &        \\
\bottomrule
\multicolumn{6}{l}{\tiny{$^{***}p<0.001$; $^{**}p<0.01$; $^{*}p<0.05$}}
\end{tabular}
\end{scriptsize}
\label{table:coefficients}
\end{center}
\end{table}


\clearpage


\pagebreak


% References here (outcomment the appropriate case)

% CASE 1: BiBTeX used to constantly update the references
%   (while the paper is being written).
 \bibliographystyle{informs2014} % outcomment this and next line in Case 1
\bibliography{DCE.bib} % if more than one, comma separated
%%  \bibliographystyle{elsarticle-num-names} 
%%  \bibliography{<your bibdatabase>}

%% else use the following coding to input the bibitems directly in the
%% TeX file.

\end{document}

\endinput
%%
%% End of file `elsarticle-template-num-names.tex'.
