

\documentclass[3p,12pt ]{elsarticle}

%% Use the option review to obtain double line spacing
%% \documentclass[preprint,review,12pt]{elsarticle}

%% Use the options 1p,twocolumn; 3p; 3p,twocolumn; 5p; or 5p,twocolumn
%% for a journal layout:
%% \documentclass[final,1p,times]{elsarticle}
%% \documentclass[final,1p,times,twocolumn]{elsarticle}
%% \documentclass[final,3p,times]{elsarticle}
%% \documentclass[final,3p,times,twocolumn]{elsarticle}
%% \documentclass[final,5p,times]{elsarticle}
%% \documentclass[final,5p,times,twocolumn]{elsarticle}

%% For including figures, graphicx.sty has been loaded in
%% elsarticle.cls. If you prefer to use the old commands
%% please give \usepackage{epsfig}
%% The amssymb package provides various useful mathematical symbols



\usepackage{natbib}
 \bibpunct[, ]{(}{)}{,}{a}{}{,}%
 \def\bibfont{\small}%
 \def\bibsep{\smallskipamount}%
 \def\bibhang{24pt}%
 \def\newblock{\ }%
 \def\BIBand{and}%


\usepackage{amssymb}
\usepackage{amsmath}
\usepackage{booktabs}
\usepackage{longtable}
\usepackage{array}
\usepackage{multirow}
\usepackage{wrapfig}
\usepackage{float}
\usepackage{colortbl}
\usepackage{pdflscape}
\usepackage{tabu}
\usepackage{threeparttable}
\usepackage{threeparttablex}
\usepackage[normalem]{ulem}
\usepackage{makecell}
\usepackage{xcolor}

\usepackage[nomarkers]{endfloat}
\usepackage{setspace}
\usepackage{graphicx}
\usepackage{float}
\usepackage{rotating}
\usepackage{hyperref}
\hypersetup{
  colorlinks=true,
  linkcolor=blue}
\usepackage{caption}
\captionsetup{font=normalsize}

%% The amsthm package provides extended theorem environments
%% \usepackage{amsthm}

%% The lineno packages adds line numbers. Start line numbering with
%% \begin{linenumbers}, end it with \end{linenumbers}. Or switch it on
%% for the whole article with \linenumbers.
%% \usepackage{lineno}

\journal{SSRN}

\begin{document}

\begin{frontmatter}

%% Title, authors and addresses

%% use the tnoteref command within \title for footnotes;
%% use the tnotetext command for theassociated footnote;
%% use the fnref command within \author or \address for footnotes;
%% use the fntext command for theassociated footnote;
%% use the corref command within \author for corresponding author footnotes;
%% use the cortext command for theassociated footnote;
%% use the ead command for the email address,
%% and the form \ead[url] for the home page:
%% \title{Title\tnoteref{label1}}
%% \tnotetext[label1]{}
%% \author{Name\corref{cor1}\fnref{label2}}
%% \ead{email address}
%% \ead[url]{home page}
%% \fntext[label2]{}
%% \cortext[cor1]{}
%% \affiliation{organization={},
%%             addressline={},
%%             city={},
%%             postcode={},
%%             state={},
%%             country={}}
%% \fntext[label3]{}

\title{The value of location: what matters most for relocating older individuals in Sweden?}

%% use optional labels to link authors explicitly to addresses:
%% \author[label1,label2]{}
%% \affiliation[label1]{organization={},
%%             addressline={},
%%             city={},
%%             postcode={},
%%             state={},
%%             country={}}
%%
%% \affiliation[label2]{organization={},
%%             addressline={},
%%             city={},
%%             postcode={},
%%             state={},
%%             country={}}

%\author[1]{Maya Ky\'eln}
%\ead{maya.kyeln@med.lu.se}

%\author[1]{Bj\"orn Slaug}
%\ead{bjorn.slaug@med.lu.se}

\author[1]{Nick Christie}
\ead{susanne.iwarsson@med.lu.se}

\author[1]{Susanne Iwarsson}
\ead{susanne.iwarsson@med.lu.se}

%\author[1]{Steven M Schmidt}
%\ead{steven.schmidt@med.lu.se}

\author[2]{Jonas Bj\"ork}
\ead{jonas.bjork@med.lu.se}

\author[1]{Magnus Zingmark}
\ead{magnus.zingmark@med.lu.se}











 \affiliation[1]{organization={Department of Health Sciences, Lund University}, 
                 addressline={P.O. Box 7080},
                 postcode={22100}, 
                 city={Lund}, 
                 country={Sweden}}
                 
 \affiliation[2]{organization={Division of Occupational and Environmental Medicine, Lund University}, 
                 addressline={Scheelevägen 2},
                 postcode={22363}, 
                 city={Lund}, 
                 country={Sweden}}
                 
                 
\begin{abstract}

Using a diverse sample of rural elderly homeowners across Sweden, this paper explores the factors influencing their housing choices. Respondents were presented with hypothetical scenarios varying in attributes such as proximity to healthcare facilities, access to public transportation, green spaces, social amenities, and natural surroundings, allowing us to assess preferences and attribute importance.
The findings reveal that access to public transportation and proximity to green spaces are the most critical factors influencing the housing choices of elderly homeowners in rural areas. These insights have practical implications for rural planners, policymakers, and healthcare providers, guiding the creation of age-friendly rural environments that meet the needs of elderly homeowners.
This research contributes to the discourse on rural sustainability and aging in place by emphasizing the significance of public transportation and green spaces in shaping the residential decisions of rural elderly homeowners. These results provide a foundation for targeted interventions and policies to enhance the quality of life for this demographic and bolster resilient rural communities.
\end{abstract}


%\begin{keyword}
%
%Ageing \sep Housing \sep Health \sep Relocation
%%% keywords here, in the form: keyword \sep keyword
%
%%% PACS codes here, in the form: \PACS code \sep code
%
%%% MSC codes here, in the form: \MSC code \sep code
%%% or \MSC[2008] code \sep code (2000 is the default)
%
%\end{keyword}
\end{frontmatter}

%% \linenumbers

%\newpage

\section{Introduction}



This study employs a discrete choice experiment (DCE) to delve into the critical factors influencing the housing choices of older indviduals looking to relocate in Sweden.

Our research uses a diverse sample individuals, presenting them with hypothetical scenarios that vary in locational attributes, including proximity to green areas, public transportation, shopping, and parking availability.
By analyzing their responses, we aim to elicit preferences and estimate the relative importance of these attributes.

Among these attributes, two factors stand out as paramount: access to public transportation and proximity to green spaces 

\section{Methods}

This study utilizes a discrete choice experiment (DCE) to investigate the locational preferences of elderly homeowners in rural environments. DCE is a quantitative research method that presents respondents with a series of hypothetical scenarios, each with varying attributes, to elicit preferences and estimate the relative importance of these attributes in decision-making (Louviere et al., 2015).

\subsection{Sampling and }

A diverse sample of rural elderly homeowners aged 65 and above was recruited for this study. Participants were selected using a combination of purposive and random sampling methods to ensure representation across different rural regions and demographic characteristics. Informed consent was obtained from all participants

\subsection{Empirical design}

In a Discrete Choice Experiment (DCE), the utility that an individual assigns to a particular choice alternative is often estimated using a random utility model. A commonly used model is the conditional logit model. The utility of an alternative is represented as:

\begin{equation}
U_{ij} = V_{ij} + \varepsilon_{ij}
\end{equation}

Where:
\begin{align*}
&U_{ij} \text{ is the total utility that respondent } i \text{ associates with choice alternative } j. \\
&V_{ij} \text{ is the systematic utility, which is a function of the attributes of the alternative and the preferences of the respondent.} \\
&\varepsilon_{ij} \text{ is the random error term, representing unobserved factors and individual-specific preferences. It is assumed to follow an independent and identically distributed extreme value Type I (Gumbel) distribution.}
\end{align*}

The systematic utility, \( V_{ij} \), is typically modeled as a linear function of attribute levels:

\begin{equation}
V_{ij} = \beta_1 X_{ij1} + \beta_2 X_{ij2} + \ldots + \beta_k X_{ijk}
\end{equation}

Where:
\begin{align*}
&\beta_1, \beta_2, \ldots, \beta_k \text{ are the coefficients representing the part-worth utilities of the attribute levels.} \\
&X_{ij1}, X_{ij2}, \ldots, X_{ijk} \text{ are the levels of the attributes for choice alternative } j \text{ in scenario } i.
\end{align*}

The probability that respondent \( i \) chooses alternative \( j \) from a set of alternatives is modeled using the choice probability as follows:

\begin{equation}
P_{ij} = \frac{e^{V_{ij}}}{\sum_{l=1}^{J} e^{V_{il}}}
\end{equation}

Where:
\begin{align*}
&P_{ij} \text{ is the probability that respondent } i \text{ chooses alternative } j. \\
&J \text{ is the total number of alternatives in the choice set.}
\end{align*}

To estimate the coefficients (\( \beta \) values) in the systematic utility equation, maximum likelihood estimation (MLE) methods are typically used. The likelihood function for the conditional logit model is maximized to find the best-fitting coefficients that maximize the probability of observing the actual choices made by respondents given the attributes of the alternatives.

The estimated coefficients (\( \hat{\beta} \)) can then be used to calculate the relative importance of each attribute and the trade-offs respondents are willing to make between attribute levels when choosing among alternatives.



\section{Conclusions}



\newpage


\pagebreak


% References here (outcomment the appropriate case)

% CASE 1: BiBTeX used to constantly update the references
%   (while the paper is being written).
 \bibliographystyle{informs2014} % outcomment this and next line in Case 1
\bibliography{lac.bib} % if more than one, comma separated
%%  \bibliographystyle{elsarticle-num-names} 
%%  \bibliography{<your bibdatabase>}

%% else use the following coding to input the bibitems directly in the
%% TeX file.

\end{document}

\endinput
%%
%% End of file `elsarticle-template-num-names.tex'.
